\newglossaryentry{b2b}
{
        name=B2B,
        text=B2B,
        description={Business-to-Business. Denominação do comércio estabelecido entre empresas}
}
\newglossaryentry{lead}
{
	name=Lead,
	text=Lead,
	description={Potencial consumidor de uma marca, que demonstrou interesse num produto ou serviço, mas que ainda não está qualificado, ou seja, não temos quaisquer informações}
}

\newglossaryentry{prospects}
{
	name=Prospects,
	text=Prospects,
	description={Potencial consumidor de uma marca, que demonstrou interesse num produto ou serviço, que está qualificado, ou seja, já temos informações sobre ele. Por outras palavras é uma \textit{lead} qualificada.}
}

\newglossaryentry{backoffice}
{
	name=BackOffice,
	text=BackOffice,
	description={Secção da plataforma feita para administração e configuração, uma espécie de painel de controle. Esta secção não é visível pelo utilizador final. }
}

\newglossaryentry{tags}
{
	name=Tags,
	text=Tags,
	description={Palavras chaves para organização e classificação de informações. }
}

\newglossaryentry{social selling}
{
	name=Social Selling,
	text=Social Selling,
	description={Processo de encontrar os prospects certos, criar relações de confiança e, idealmente, atingir os objetivos de vendas da empresa.}
}

\newglossaryentry{webinars}
{
	name=Webinars,
	text=Webinars,
	description={Abreviação de \textit{Web-based seminar}. Um webinar é um \textit{workshop}, aula, seminário ou apresentação, em formato de vídeo conferência, transmitido através da \textit{web}.}
}

\newglossaryentry{user}
{
	name=Utilizador,
	text=Utilizador,
	description={Utilizador da plataforma em questão e/ou cliente (empresa).}
}

\newglossaryentry{finaluser}
{
	name=Utilizador Final,
	text=Utilizador Final,
	description={Prospects, Leads, Clientes ou Participantes de uma formação, concurso ou questionário.}
}

\newglossaryentry{pop}
{
	name=\textit{Project on a Page},
	text=\textit{Project on a Page},
	description={Modelo de negócio descrito numa página}
}


