\chapter{Abordagem}
\label{sec:abordagem}

Este capítulo tem como principal objectivo descrever a metodologia adotada e ainda apresentar o planeamento do projeto tal como os desvios relativamente ao mesmo. Por fim será apresentada uma análise dos riscos associados ao projeto que poderão ter um impacto negativo no plano de desenvolvimento. Este conjunto de passos foca-se em atingir um produto final bem estruturado e funcional, utilizando boas práticas de desenvolvimento de software


\section{Metodologia}
\label{metodologia}

A metodologia de desenvolvimento do projeto adoptada foi uma metodologia fortemente baseada em SCRUM\cite{scrum}, que será descrita (i. e. os aspectos mais importantes) nesta secção. Esta é a metodologia utilizada pela 10.digital e tendo em conta que esta metodologia ágil se encaixa perfeitamente nas necessidades do projeto, estes foram os fatores decisivos para a escolha da mesma. 

Líder em desenvolvimento ágil, o SCRUM, é uma metodologia apontada para projetos com foco em trazer valor ao cliente de forma incremental, através de iterações de curta duração, chamadas \textit{Sprints}. Esta metodologia possibilita também a abordagem de problemas complexos de forma produtiva, priorizar tarefas durante a fase de desenvolvimento e ainda facilita a inclusão de novas funcionalidades sempre que necessário. Desta forma o SCRUM proporciona uma gestão flexível do projeto e permite realizar pequenas alterações no planeamento, sem necessidade de interromper o desenvolvimento.

\subsection{Intervenientes}

Um aspecto determinante para o sucesso do SCRUM é o trabalho em equipa. Tipicamente nesta metodologia existem três papéis pré-definidos: \textit{Product Owner}, \textit{Scrum Master} e \textit{Scrum Team}.

O \textbf{\textit{Product Owner}} representa o cliente e é responsável por transmitir a visão do produto, ou por outras palavras, é responsável por maximizar o valor do produto e o trabalho da equipa de desenvolvimento (i. e. \textit{Scrum Master}). É responsável por organizar e priorizar as tarefas no \textit{product backlog}.

O \textbf{\textit{Scrum Master}} tem um papel fundamental no desempenho da \textit{Scrum Team}. É responsável por garantir o cumprimento das práticas do SCRUM, ajudar e orientar a \textit{Scrum Team} especialmente nas dificuldades que vão surgindo ao longo do projeto e, de forma gradual (i. e. respeitando as \textit{Sprints}), apresentar o trabalho realizado ao \textit{Product Owner}.

A \textbf{\textit{Scrum Team}} representa os elementos que constituem a equipa de desenvolvimento que com a orientação do \textit{Scrum Master}, monitorizam o trabalho que vai sendo feito e assim conseguir cumprir com as \textit{sprints backlog}. A \textit{Scrum Team} deve ser autónoma e organizada. 

\subsection{Processo}

O Desenvolvimento começa assim que o \textit{product backlog} estiver concluído e detalhado pelo \textit{Product Owner}. O \textit{product backlog} representa a lista requisitos necessários para atingir o produto final. 

Depois de definido o \textit{product backlog}, o \textit{Scrum Master}, juntamente com a \textit{Scrum Team} reúnem e definem o tempo para cada \textit{Sprint}. Tipicamente as \textit{Sprints} tem duração entre 1 a 4 semanas e no final das mesmas é apresentado o trabalho realizado pela \textit{Scrum Team}. No início de cada \textit{Sprint} é criada a \textit{Sprint Backlog}. Na \textit{Sprint Backlog} é estipulado o conjunto de funcionalidades/tarefas a realizar durante a \textit{Sprint}. Na 10.digital a variável da velocidade não é implementada na \textit{Sprint Backlog}.

No início de cada dia é realizada a \textit{Daily Scrum}, uma reunião que tem uma duração de 15$\pm$5 minutos, onde, de forma informal, são discutidas as tarefas que devem ser implementadas nesse dia, o ponto de situação do projeto relativo ao dia anterior e caso haja algum impasse ou dificuldade na realização de alguma tarefa, imediatamente após a reunião (i. e. assim que possível) tentasse arranjar uma solução para o mesmo. 

No final de cada \textit{Sprint} há uma reunião (\textit{Sprint Review}) para verificar as tarefas que foram realizadas. Durante a reunião a \textit{Scrum Team} apresenta as novas funcionalidades implementadas para os restantes participantes que podem ser \textit{Product Owner}, \textit{Scrum Master}, clientes e outros colegas de trabalho. 

A Figura \ref{fig:scrum} sumariza todo o processo descrito anteriormente.



\begin{figure}[ht!]
	\begin{center}
		\includegraphics[width=1\textwidth]{img/scrum.pdf}
		\caption{Scrum Framework\cite{scrumimg}}
		\label{fig:scrum}
	\end{center}
\end{figure}

\newpage

\section{Planeamento}
\label{planeamento}

Nesta secção será apresentado o plano de estágio do primeiro e segundo semestre, através de diagramas de Gantt.  De seguida será exposto o desvio temporal em relação ao planeado e por fim serão brevemente detalhados os artefactos (i. e. \textit{Sprint Backlog}) da metodologia SCRUM.

Como foi referido no Capítulo \ref{sec:introducao}, este documento expõe o trabalho realizado neste projeto durante o ano lectivo, por isso, apenas será exposto o desenvolvimento do \textit{back-end} da plataforma de inbound marketing. Dito isto, a equipa de desenvolvimento será composta por múltiplas pessoas sendo que cada um terá a sua função distinta.
No seguimento do ponto anterior, os cargos de cada interveniente no projeto são:
\begin{itemize}
	\item[--] \textbf{\textit{Product Owner}}: Eng. Pedro Beck (\acrfull{cto})
	\item[--] \textbf{\textit{Scrum Master}}: Mário Melo (Coordenador de Projetos)
	\item[--] \textbf{\textit{Scrum Team}}: 
	\subitem  \textit{Front-End Developer} - Por Definir
	\subitem  \textit{Back-End Developer} - Bruno Grifo
	\subitem  \textit{Back-End Developer \acrshort{tcg}} - Eng. Pedro Beck
\end{itemize}

É ainda de referir que, no primeiro semestre, o Mestre João Oliveira realizou o papel de co-orientador na empresa e teve um impacto importante na orientação do projeto de tese. 

\begin{figure}[ht!]
	\begin{center}
		\includegraphics[width=1\textwidth]{img/gantt/semestre1.jpeg}
		\caption{Diagrama de Gantt - Planeamento do 1º semestre}
		\label{fig:gantt1}
	\end{center}
\end{figure}

Representado na Figura \ref{fig:gantt1} temos o plano de estágio relativo ao primeiro semestre. A escrita do relatório intermédio foi dividido nas seguintes 6 tarefas principais:
\begin{itemize}
	\item \textbf{Contextualização Inbound Marketing e Introdução}: Tendo em conta a àrea onde o projeto se ínsere, foi realizado um estudo sobre inbound marketing para que me pudesse contextualizar e conseguir compreender as diversas estratégias de inbound. No final da tarefa foi realizada uma apresentação sobre este tema e avaliada pelo co-orientador da empresa, para garantir o nível de conhecimento pretendido. Este estudo foi crucial para desenhar o modelo de negócio do projeto.
	\item \textbf{Estado de Arte}: Nesta tarefa foi feito o levantamento do estado de arte. Foram analisadas várias aplicações/plataformas concorrentes ou com funcionalidades semelhantes para ganhar um melhor conhecimento sobre o mercado.
	\item \textbf{Metodologia e Planeamento}: Foi feito um estudo interno para perceber ao detalhe como foi adoptada a metodologia SCRUM na empresa.
	\item \textbf{Especificação de Requisitos}: Esta tarefa começou com a elaboração de alguns protótipos de baixa fidelidade e um conjunto de requisitos funcionais. De seguida foi marcada uma reunião com o cliente onde, pegando no trabalho realizado anteriormente, foram feitos os devidos ajustes e levantado os restantes requisitos. Foram assim documentados os requisitos não funcionais, funcionais e respetivos casos de uso e ainda as restrições técnicas e de negócios.
	\item \textbf{Prototipagem de produto de baixa fidelidade}: Foi criado um conjunto de protótipos que representam todas as funcionalidades principais da plataforma e que de igual forma satisfazem o caso de uso correspondente.
	\item \textbf{Arquitectura de Software}: Nesta tarefa foi projetada a arquitectura a desenvolver no âmbito do estágio.
\end{itemize}

Neste projeto cada \textit{sprint} dura duas semanas e a \textit{sprint meeting} é feita no último dia. Em cada\textit{ sprint meeting }está presente a equipa de desenvolvimento (\textit{scrum team}), os orientadores (\textit{product owner}) e, sempre que possível, o tutor. Cada estagiário apresenta o que fez durante o \textit{sprint}, executa a \textit{demo} das funcionalidades implementadas e partilha as dificuldades encontradas durante o \textit{sprint }com os restantes. Os orientadores dão \textit{feedback}, tanto do resultado geral do \textit{sprint}, como de cada funcionalidade (\textit{user story}) implementada.


Neste semestre cada \textit{Sprint} tem a duração de 2 semanas e no final da mesma é feita uma reunião de ponto (i. e. \textit{Daily Scrum}), onde estão presentes o \textit{Product Owner}, \textit{Scrum Master}, clientes e outros colegas de trabalho que queiram participar. Nesta primeira fase, no total foram realizadas 7 \textit{Sprints}:

\begin{itemize}
	\item \textbf{\textit{Sprint} \#1}
		\subitem \textbf{Data Início}: 16/09/2019
		\subitem \textbf{Data Fim}: 27/09/2019
		\subitem \textbf{Descrição}: Estudo sobre Inbound Marketing.
	\item \textbf{\textit{Sprint} \#2}
		\subitem \textbf{Data Início}: 28/09/2019
		\subitem \textbf{Data Fim}: 11/10/2019
		\subitem \textbf{Descrição}: Escrita do Capítulo \ref{sec:introducao} e início do estudo sobre aplicações/plataformas concorrentes.
	\item \textbf{\textit{Sprint} \#3}
		\subitem \textbf{Data Início}: 12/10/2019
		\subitem \textbf{Data Fim}: 25/10/2019
		\subitem \textbf{Descrição}: Conclusão do estudo de mercado e escrita do Capítulo \ref{sec:estado-arte}.
	\item \textbf{\textit{Sprint} \#4}
		\subitem \textbf{Data Início}: 26/10/2019
		\subitem \textbf{Data Fim}: 08/11/2019
		\subitem \textbf{Descrição}: Escrita do Capítulo \ref{sec:abordagem} e reunião com o cliente para levantamento de requisitos.
	\item \textbf{\textit{Sprint} \#5}
		\subitem \textbf{Data Início}: 09/11/2019
		\subitem \textbf{Data Fim}: 22/11/2019
		\subitem \textbf{Descrição}: Escrita do Capítulo \ref{sec:requisitos}.
	\item \textbf{\textit{Sprint} \#6}
		\subitem \textbf{Data Início}: 23/11/2019
		\subitem \textbf{Data Fim}: 06/12/2019
		\subitem \textbf{Descrição}:  Escrita dos casos de Uso no Anexo \ref{a:cu} e prototipagem do produto de baixa fidelidade no Anexo \ref{a:prototipos}.
	\item \textbf{\textit{Sprint} \#7}
		\subitem \textbf{Data Início}: 06/11/2019
		\subitem \textbf{Data Fim}: 20/12/2019
		\subitem \textbf{Descrição}: Projeção da arquitectura e escrita do Capítulo \ref{sec:arquitetura}
\end{itemize}

\begin{figure}[ht!]
	\begin{center}
		\includegraphics[width=1\textwidth]{img/gantt/semestre2.jpeg}
		\caption{Diagrama de Gantt - Planeamento do 2º semestre}
		\label{fig:gantt2}
	\end{center}
\end{figure}

Representado na Figura \ref{fig:gantt2} temos o plano de estágio relativo ao segundo semestre. A fase de implementação divide-se em várias \textit{Sprints}, que serão detalhadas no Capítulo \ref{sec:sprints}.

\newpage

\begin{figure}[ht!]
	\begin{center}
		\includegraphics[width=1\textwidth]{img/gantt/vs.jpeg}
		\caption{Diagrama de Gantt - Planeamento vs Real do 1º semestre}
		\label{fig:ganttvs1}
	\end{center}
\end{figure}

Representado na Figura \ref{fig:ganttvs1} temos o planeamento de estágio relativo ao primeiro semestre em comparação com o tempo que o aluno realmente demorou a concluir as tarefas. A primeira tarefa que salta à vista é o Estado de Arte, que claramente excedeu a data limite de entrega para a reunião de ponto com os orientadores de estágio. Um dos fatores decisivos para o atraso na realização da tarefa deveu-se ao facto de ser uma fase de pesquisa e análise do mercado atual, em que é difícil de prever o tempo exato que se demora a realizar esta tarefa. O outro fator decisivo deveu-se ao facto de, a meio do semestre, o cliente ter pedido para implementar uma nova funcionalidade que representa um dos principais objetivos na plataforma a desenvolver. Tudo isto criou um atraso significativo em todas as restantes tarefas. Podemos também reparar que a escrita da tarefa da Metodologia e Planeamento foi realizada algumas semanas depois do planeado. O aparecimento da nova funcionalidade levou a uma reestruturação do planeamento o que provocou um atraso nesta tarefa, visto que só se inicializou a escrita da mesma depois de ter acabar o novo planeamento.



\glsresetall



