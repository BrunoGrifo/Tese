\chapter{Termos de Serviço}
\label{sec:gdpr}

Estes termos de serviço regulam as condições de acesso e uso da plataforma 10.Quest. 

Na Dascat Software Lda., a empresa que desenvolveu o 10.Quest, respeitamos a privacidade de quem utiliza os nossos produtos e serviços, entre os quais os sites e aplicações que disponibilizamos. Nesta secção referimo-nos à Dascat Software Lda. como “Dascat Software”, “nós” e “nosso”; ao utilizador como "utilizador", “você”, “lhe”, “si” ou “seu”; e ao participante das campanhas online e/ou passatempos como "participante".

O 10.Quest é uma plataforma online desenvolvida pela Dascat Software com sede na no Instituto Pedro Nunes, Ipn Incubadora (Edifício C), Rua Pedro Nunes 3030-181 Coimbra.

A Dascat Software Lda. reserva-se do direito de atualizar os presentes Termos de Serviço, sem a obrigação de notificar seus utilizadores quando essas alterações ocorrerem. No entanto, e apesar de nós enviarmos um mail sempre que sejam feitas atualizações, aconselhamos-lhe a consultar os termos e condições regularmente.

\section{Descrição do Serviço}

Tal como referido no Capítulo XX o 10.Quest é uma plataforma online que permite aos utilizadores criar campanhas online. Os dados recolhidos através destas campanhas são tratados, permitindo ao utilizador segmentar leads por perfis e assim optimizar outras campanhas e conteúdos que se fazem chegar aos mesmos.
O utilizador é responsável pela gestão da sua conta da plataforma, isto é, adição, edição e remoção de conteúdo necessário para a criação de campanhas (perguntas, respostas, tags, imagens, áudios e vídeos).

\section{Responsabilidades}

O 10.quest recolhe, através de campanhas online, informações dos participantes. Para salvaguardar a privacidade dos utilizadores e participantes e seguir as normas do RGPD (Regulamento Geral de Proteção de Dados)\cite{f10}, foi adotada uma política de privacidade para a garantir que os dados são tratados e armazenados de forma segura.
Expandindo o que foi mencionado anteriormente, os dados armazenados pela plataforma são os seguintes:
\begin{itemize}
	\item Dados de registo na plataforma da 10.digital.
	\item Dados relativos à criação e gestão de campanhas na plataforma.
	\item Dados recolhidos no âmbito da participação em campanhas online.
\end{itemize}

Estes dados serão guardados pela plataforma até que o utilizador deseje apagar a sua conta. O consentimento será solicitado na criação de uma nova conta na plataforma e antes da participação numa campanha online. Sem o consentimento expresso do utilizador ou participante, não será possível criar um registo na plataforma nem participar numa campanha online, respectivamente. Toda a informação recolhida não será vendida a terceiros e não será utilizada para nenhum propósito que não vá ao encontro dos objectivos do 10.Quest.

\section{Responsável pela Proteção dos Dados}

Uma das funcionalidades chave da plataforma 10.Quest é a análise e representação dos dados recolhidos nas campanhas online e passatempos. Apesar de não se tratar de informação sensível, nem processamento de dados a uma larga escala, a plataforma 10.Quest requer um responsável pela proteção dos dados de forma a cumprir com o RGPD. O responsável pela proteção dos dados é o Engenheiro Pedro Beck, CTO na 10.digital, que será responsável por monitorizar as actividades da plataforma de forma a garantir que a mesma cumpre com o RGPD e outras leis de proteção de dados existentes nos países em que opera.

\section{Política de Privacidade}

\subsection{Introdução}

Nesta secção, a “Política de Privacidade”, explica como recolhemos, processamos e armazenamos qualquer dado pessoal que lhe pertença, ou seja, informação que permite identificá-lo, como o nome, a morada, o número de identificação fiscal ou o email. Esta informação é referida neste documento como “informação pessoal” ou “dados pessoais”.

\subsection{Informação que recolhemos, como recolhemos e como a utilizamos}

\subsubsection{Registo na Plataforma}

O acesso à plataforma pode ser efetuado a qualquer hora. Para que tal seja possível é necessário que o utilizador tenha uma conta registada. Se estiver registado, pode aceder com o email e password. Esta informação pode ser temporariamente armazenada pelos cookies que utilizamos para permitir que numa próxima visita à plataforma não tenha de voltar a realizar este processo.

Durante a fase Beta da plataforma, o utilizador deve realizar um pedido de registo para aceder ao serviço. Para tal o utilizador deve preencher o formulário no nosso website e esperar pelo nosso contacto. A equipa da Dascat Software reserva o direito de restringir e selecionar o registo dos utilizadores.

\subsubsection{Campanhas Online}

Para ser possível participar numa campanha online, é solicitado o consentimento do participante e o mesmo tem que concordar com os termos e condições, descritos nesta secção.
Alguns dados pessoais são recolhidos, que identificam os participantes, como o nome e endereço de email. Os dados recolhidos das campanhas serão também guardados e processados de forma a conseguir segmentar os participantes e conseguir criar perfis. É ainda importante referir que as informações pessoais dos participantes podem ser usadas para serem contactados para fins de marketing.
Sempre que necessário, solicitaremos o seu consentimento para que possamos utilizar os seus dados para outras finalidades.

\subsubsection{Redes Sociais}

Utilizamos as redes sociais para divulgação das campanhas, caso o utilizador deseje. A sua relação com as redes sociais é regulada pelos termos e condições dessas mesmas redes. Poderá também informar-se sobre a forma como as diferentes redes sociais protegem os seus dados pessoais nas respetivas políticas de privacidade das mesmas.

\subsubsection{TheCompanyGym}

O TheCompanyGym é uma plataforma de treino online, desenvolvida pela Dascat Software, que permite aos utilizadores enviar formações curtas e simples com uma aprendizagem baseada em tentativa erro. Neste sentido, todas as funcionalidades de gestão de campanhas do tipo formação são utilizadas através de uma API de comunicação ao TCG. Todos os dados relativos às campanhas do tipo formações, inclusivo respostas dos participantes às campanhas online do tipo formação, são guardados nas bases de dados do TCG. Pode consultar a política de privacidade do TCG aqui: \url{https://www.thecompanygym.com/termspolicy.html}

\subsection{Vamos partilhar a informação com terceiros?}

A Dascat Software subcontrata terceiros para o tratamento de alguns dados pessoais recolhidos. As empresas que contratamos são cuidadosamente escolhidas de forma a assegurar que têm implementados os procedimentos necessários ao cumprimento da legislação sobre protecção de dados.

Em seguida estão listados os motivos pela qual a Dascat Software subcontrata as empresas:

\begin{itemize}
	\item[--] \textbf{Sendgrid Inc.} - Empresa que fornece o serviço de envio de emails para diferentes funcionalidades no 10.Quest, como por exemplo a confirmação do registo, recuperação de passwords, notificações de campanhas e outro tipo de comunicações automáticas via email. As comunicações de email processadas pelo Sendgrid são automáticas, sendo que esta subcontratante tem acesso aos endereços de email para onde o nosso sistema requer estes envios, bem como o conteúdo do email. A transmissão de dados para o Sendgrid (como o endereço de email) é feita por canais encriptados e seguindo todas as melhores práticas. Pode ler mais sobre a política de privacidade do Sendgrid aqui: \url{https://www.twilio.com/legal/privacy}
	\item[--] \textbf{Stripe} - POR DEFINIR \url{https://stripe.com/en-pt/privacy}
	\item[--] \textbf{Facebook Ireland Limited} - Empresa detentora de redes sociais (Facebook.com e Instagram.com) utilizada pela plataforma para divulgação. A sua relação com o Facebook está regulado pela política de privacidade das próprias plataformas. Pode ler mais sobre as suas políticas de privacidade aqui: \url{https://www.facebook.com/privacy/explanation} e \url{https://help.instagram.com/155833707900388}
	\item[--] \textbf{Twitter, Inc.} - Empresa detentora da rede social Twitter, utilizada pela plataforma para divulgação. A sua relação com o Twitter está regulado pela política de privacidade das próprias plataformas. Pode ler mais sobre a política de privacidade do Twitter aqui: \url{https://twitter.com/pt/privacy}
\end{itemize}

\subsection{Como protegemos a sua informação pessoal?}

Implementamos as medidas de segurança necessárias destinadas a assegurar a confidencialidade dos dados pessoais que recolhemos e processamos, bem como a prevenir a sua difusão não autorizada, acessos não autorizados, perdas ou a destruição de dados pessoais. Importa, contudo, salientar que nenhum método de segurança ou de encriptação pode garantir total proteção de informação contra hackers ou erro humano.

\subsection{Durante quanto tempo retemos a sua informação?}

Nós guardamos a informação recolhida por um período de tempo indefinido.

\subsection{Os seus direitos em relação à sua informação}

Poderá exercer os seus direitos de acesso, rectificação, eliminação, limitação e/ ou portabilidade através de comunicação escrita, acompanhada de documento que comprove a sua identidade e/ou os dados a retificar, se aplicável, que deve ser enviada para o e-mail: \texttt{geral@10.digital}

A oposição ao tratamento de dados aqui referido que seja comunicado à Dascat Software após o início do mesmo terá efeitos apenas a partir da data de recepção de tal comunicação, não afectando a legitimidade do tratamento até aí efectuado.

\subsection{Como contactar-nos}

Se tiver alguma questão ou preocupação sobre a nossa política de privacidade, por favor contacte-nos para: \texttt{geral@10.digital}

Instituto Pedro Nunes, Ipn Incubadora (Edifício C), Rua Pedro Nunes 
3030-181 Coimbra.

\subsection{Reclamações}

Esperamos que não tenha qualquer queixa sobre nós ou o nosso serviço. No entanto, se estiver insatisfeito com o nosso uso da sua informação pessoal por favor contacte-nos para: \texttt{geral@10.digital}

Pode ainda apresentar uma queixa à Comissão Nacional de Protecção de Dados (CNPD) caso entenda que os tratamentos de dados que realizamos não são conformes com a legislação aplicável.
