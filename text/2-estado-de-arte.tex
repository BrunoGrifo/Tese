\chapter{Estado de Arte}
\label{sec:estado-arte}

Neste capítulo será feita a comparação de algumas plataformas existentes para recolha de dados com recurso a estratégias de inbound marketing. Será também feito um resumo das funcionalidades do \acrshort{tcg}, que apesar de já estar no mercado, será integrado neste projeto. Tendo em conta que a plataforma a desenvolver irá utilizar as funcionalidades para criação de formações do \acrshort{tcg} e que a equipa do \acrlong{tcg} já fez o estudo de mercado antes do desenvolvimento do mesmo, e continua a fazer todos os dias, a análise de plataformas focadas em criação de formações não será feita neste documento. Uma análise mais detalhadas de todas as plataformas estudadas encontra-se no Anexo \ref{a:ea}.

A recolha de informação nos dias de hoje tem um grande impacto na forma como os negócios são feitos, principalmente na internet.

Como referido no capítulo anterior, o objectivo deste estágio incide na criação de uma plataforma de inbound marketing que tem como principais objectivos conseguir criar  questionários, formações,concursos e analisar, filtrar e segmentar os dados recolhidos. Neste sentido serão analisadas algumas das soluções existentes que, apesar de algumas terem um propósito distinto, podem ser utilizadas em estratégias de inbound marketing e partilham funcionalidades semelhantes com o que vai ser desenvolvido. 

A recolha de informação nos dias de hoje tem um grande impacto na forma como os negócios são feitos, principalmente na internet. Seguindo uma das estratégias de inbound marketing, o método de recolha de dados será através de concursos, questionários e formações online e por isso mesmo algumas características associadas à experiência do utilizador, como por exemplo a personalização dos mesmos, serão também analisadas.

%Rever o que está para baixo

As plataformas \textit{SurveyMonkey}\cite{surveymonkey}, \textit{Typeform}\cite{typeform}, \textit{Google Forms}\cite{googleform} são plataformas de criação de formulários, mas apesar de servirem um propósito distinto ao da plataforma a desenvolver, partilham funcionalidades semelhantes com as que vão ser desenvolvidas e por isso mesmo serão analisadas nesse sentido. Será também exposto o funcionamento do \acrshort{tcg} para melhor entendimento de como podem ser implementadas as funcionalidades e a \acrshort{api} do lado da plataforma a desenvolver. A integração do \acrshort{tcg} neste projeto, foi uma escolha da empresa (i. e. cliente), e por isso não serão analisadas plataformas concorrentes do \acrshort{tcg} na medida em que esse trabalho é feito pela equipa do \acrshort{tcg} e apenas serão feitas as mudanças necessárias no mesmo para o desenvolvimento da \acrshort{api} de comunicação. Por último serão analisadas as plataformas \textit{involve.me}\cite{involve}, \textit{Survey Anyplace}\cite{surveyA} e \textit{Interact}\cite{interact} que são algumas das serviços que podem concorrer directamente com o 10.quest (i. e. plataforma a desenvolver).

Após a apresentação destas ferramentas será feita uma análise das vantagens e desvantagens de cada uma, assim como a comparação de funcionalidades.

\section{Plataformas de criação de formulários}
\label{formulários}

\subsection{SurveyMonkey}
\label{surveyMonkeyM}

O SurveyMonkey é uma plataforma \acrfull{saas} de criação de formulários online para estudo de mercado, análise de competidores, \textit{feedback} de clientes e colaboradores, entre outros. Permite recolher informações do público alvo através de formulários e personalizar, segmentar e visualizar esta informação.

As várias funcionalidades do SurveyMonkey foram desenhadas para ajudar os utilizadores a criar diferentes tipos de formulários e recolher os resultados em tempo real. Estes dados são depois analisados e reportados através de funcionalidades próprias do SurveyMonkey. Estes dados podem ainda ser exportados para satisfazer outras necessidades. 

\subsection{Typeform}
\label{typeformM}

O Typeform é uma plataforma \acrshort{saas} de criação de formulários online. É uma empresa que afirma resolver o problema dos formulários e inquéritos aborrecidos e tem também como proposta de valor o facto de conseguir criar formulários e inquéritos sem ter que programar uma única linha de código. Esta plataforma permite recolher informações do público alvo através de formulários e inquéritos personalizados, podendo no final visualizar estes dados. 


\subsection{Google Form}
\label{googleformM}

O \textit{Google Form} é uma aplicação de administração de inquéritos que está incluída no Google Drive office juntamente com o \textit{Google Docs}\cite{gdocs}, \textit{Google Sheets} e \textit{Google Slides}\cite{gslides}. Esta ferramenta permite recolher informações do público alvo através de formulários e inquéritos personalizados e automaticamente exportar os dados para uma \textit{google sheet}.

\subsection{Análise de concorrentes indirectos}

Nesta secção serão analisadas e comparadas três das ferramentas líderes na criação de formulários online, listadas anteriormente. Como foi referido anteriormente, apesar do \textit{SurveyMonkey}, \textit{Typeform} e \textit{Google Form} serem plataformas que não concorrem com 10.quest e têm um propósito distinto, partilham funcionalidades semelhantes com a plataforma que vai ser desenvolvida. Dito isto é importante perceber quais são estas funcionalidades e tentar perceber como as podemos aproveitar ou melhorar.

Como foi referido na análise presente no anexo \ref{a:ea} em todas as plataformas/ferramentas é necessário criar uma conta para aceder a todas as funcionalidades e em todas as plataformas analisadas é possível criar conta e iniciar sessão através de sistemas externos (\acrshort{api}). O Projeto a desenvolver segue um modelo \gls{b2b} e por isso mesmo não é de grande importância implementar esse tipo de funcionalidades.

O plataforma da \textit{Google} fornece, no plano gratuito, todas as funcionalidades, ao contrário do \textit{SurveyMonkey} e do \textit{Typeform}, que, tal como se passará com a plataforma a desenvolver, para se ter acesso a todas as funcionalidades, ou pacotes de funcionalidades, terá de ser paga uma subscrição.

Todas as plataformas permitem a criação de formulários do zero, e a plataforma da 10.digital não é exceção. Tal como foi definido na estratégia de negócio, os conteúdos que serão lançados nas formações, questionários e concursos não serão da autoridade da 10.digital, a menos que estejam incluídos em algum projeto relacionado. Dito isto facilmente se decidiu que a plataforma a desenvolver terá, tal como todas as outras ferramentas analisadas, a funcionalidade de poder adicionar conteúdo previamente feito de forma rápida (i. e. importar um ficheiro \textit{CSV} com todo o conteúdo estruturado). 


Tal como foi referido no capítulo \ref{sec:introducao}, secção \ref{subsec:contexto}, o inbound é uma estratégia de marketing que se foca em criar razões para os público alvo vir até nós através da criação de conteúdo interessante, útil, relevante etc... Para manter esta procura por parte dos clientes é necessário haver valor ao longo da jornada e idealmente proporcionar uma boa experiência ao utilizador. Nesta medida a personalização das formações, questionários e concursos é muito importante tanto a nível de conteúdo, como estético e funcional para que o utilizador se sinta valorizado. Para tornar isto possível, tal como o \textit{SurveyMonkey}, \textit{Typeform} e \textit{Google Form}, a plataforma a desenvolver incluirá funcionalidades que lhe permitirão personalizar a \textit{interface} para melhorar a experiência do utilizador final. 

Como é visível na análise expressa no anexo\ref{a:ea} o \textit{SurveyMonkey} e o \textit{Typeform} integram a funcionalidade de criação de um fluxo lógico. Contudo esta funcionalidade será abordada mais à frente nesta dissertação, com o intuito de fazer uma análise  mais aprofundada e comparativa com as plataformas directamente concorrentes.

Antes de enviar/partilhar um formulário é sempre importante pré-visualizar e testar. Neste aspecto a plataforma a desenvolver não é diferente. Será possível pré-visualizar as formações, questionários e concursos para verificar e validar os mesmos. 

Por último temos a análise e tratamento de dados que é um suporte fundamental ao marketing digital. A plataforma a desenvolver, a par com todas as restantes plataformas analisadas, não é um \textit{software} dedicado a esse fim, na medida que terá limitações (i. e. apenas implementa um conjunto de funcionalidades principais de análise de dados para satisfazer as necessidades do utilizador. Funcionalidades como a remoção de \textit{outliers}, calculo da tendência/previsão dos dados etc..., que envolve uma tratamento mais cuidado e por vezes não linear, não são implementadas), contudo, terá as funcionalidades necessárias (i. e. participação dos utilizadores finais, desempenho dos utilizadores finais, calculo das questões mais dificeis baseado nas respostas erradas etc...) para satisfazer as necessidades do utilizador. A plataforma \textit{SurveyMonkey}, neste aspecto, foi a única ferramenta que apresentou funcionalidades para além da análise básica conseguindo, de forma elementar, segmentar e comparar resultados, em contraste com o \textit{Typeform} e do \textit{Google Form} que apenas apresentam os resultados globais e por pergunta.

\section{\acrfull{tcg}}
\label{sec:TCGM}

O \acrlong{tcg} é um \acrshort{saas} presente atualmente no mercado, desenvolvido pela equipa da 10.digital, que tem como principal objectivo transmitir conhecimento através de uma técnica de aprendizagem baseada em tentativa e erro.


 O \acrshort{tcg} nasceu de uma forte convicção de que perder apenas 2 minutos por dia numa formação com uma abordagem tentativa e erro é uma ótima forma de aprender, poupando tempo e dinheiro às empresas. Inicialmente muito focado em formação interna, a equipa do \acrshort{tcg} foi-se apercebendo que existem muitos outros problemas (e. g. Consolidação de procedimentos, \textit{Onboarding} de novos colaboradores, Divulgação da cultura da empresa, Divulgação de informações técnicas a parceiros/clientes ...) para o qual a plataforma tem solução (e. g. Assimilação da cultura de empresa e do espírito das marcas, Simplificação do processo de acolhimento, Redução de custos em reuniões periódicas, Facilidade em divulgar aspectos técnicos, que de outra, forma demorariam mais tempo ...).\cite{tcginfo}. 
 
 As principais funcionalidades do  \acrshort{tcg}  são as seguintes:


\begin{itemize}
	\item[--] Criação de formações 
	\item[--] Tutoriais para as funcionalidades chave
	\item[--] Algoritmo de gestão automática (i. e. sistema identifica as perguntas mais difíceis baseado nos resultados do utilizador final. As perguntas mais difíceis saem com mais frequência e perguntas erradas saem no dia seguinte)
	\item[--] \textit{Feedback }dos utilizadores em cada pergunta
	\item[--] \textit{Gamification}
	\item[--] Análise detalhada dos resultados
	\item[--] Compatível com dispositivos móveis
\end{itemize}

\section{Análise de concorrentes directos}

Um dos objetivos do projeto é permitir aos utilizadores da plataforma a criação de questionários que, baseando-se nas respostas do utilizador final, apresenta um resultado  segmentado no fim do mesmo, tal como referido no Capítulo \ref{subsec:objetivos}.

O Akinator\cite{akinator} e o high5test\cite{5} são dois bons exemplos de aplicações que através de um questionário, e baseado nas respostas do utilizador final, apresenta um resultado segmentado. Estas duas aplicações são bastante poderosas e utilizam algoritmos de decisão que os permitem tirar este tipo de conclusões. Nesta medida alguns algoritmos como o CART\cite{cart}, ID3\cite{id3}\cite{id3_2}\cite{cart} e C4.5\cite{cart}\cite{c4.5} foram brevemente estudados com intuito de avaliar a viabilidade de serem aplicados na conceção desta funcionalidade do projeto. Este tipo de algoritmos, não só são complexos como são algoritmos que necessitam de dados de treino. Tal como referido no Capítulo \ref{subsec:objetivos}, um dos requisitos definidos pelo cliente é que a criação destes questionário seja intuitiva e exequível por qualquer pessoa mesmo que esta não tenha qualquer conhecimento em programação. Isto agregado ao facto de que, em qualquer momento da criação dos questionário, não haverá dados de treino, a aplicação deste tipo de algoritmos não é viável.

Dito isto foram discutidas outras abordagens e determinou-se que uma solução que vai de encontro ao conceito da funcionalidade passa pela criação de um sistema de pontuações
Nesta abordagem são atribuídos pesos a cada pergunta e consoante a resposta do utilizador, o sistema vai pontuando os resultados possíveis para que no final possa ser apresentado o resultado com maior pontuação.

Uma aplicação deste tipo de questionário será, por exemplo: "Calcule o lugar ideal de para viajar nas suas férias". Atualmente, num formato semelhante (i. e. em formato \textit{quiz}) já existem alguns \textit{Websites} (e. g. Chase for Adventure\cite{chaseforadventure}, travelpicker\cite{travelpicker}, Insight Vacations\cite{insightvacations} e Driftwood Journals\cite{driftwoodjournals}) que satisfazem, de forma simplista, esta necessidade em específico. Apesar de simples são exemplos que se conseguirião replicar com a plataforma a desenvolver e que está alinhado com a visão da plataforma. Neste sentido foram analisadas plataformas mais poderosas, que se focam na construção de leads, e que conseguem o mesmo resultado.


\subsection{involve.me}
\label{involvemeM}

%Não aceita multiplos resultados por resposta
%não cria perfis de utilizador
%multichannel supporte
%landiung page

"\textit{involve.me is a next-generation user engagement \& customer experience platform with a focus on digital marketers \& e-commerce.}"\cite{involve}. \textit{O involve.me} é uma plataforma moderna que ajuda empresas a criar interações personalizadas ao longo da jornada dos clientes, aumentando a audiência e recolhendo mais e melhores dados. Esta plataforma foca-se também na recolha e análise destes dados/informações sobre os utilizadores finais.


\subsection{Survey Anyplace}
\label{surveyanyplaceM}


%nao avisa que está live o questionário
%feature de ajuda no canto inferior direito -  multi channel


O Survey Anyplace é uma plataforma online com foco na criação de \textit{surveys} e questionários interativos. O Survey Anyplace afirma proporcionar uma boa experiência para o utilizador facultando-lhe elementos interativos e funcionalidades de personalização. Esta plataforma permite também a análise dos dados recolhidos através dos questionários publicados.


\subsection{Interact}
\label{interactM}


O Interact é uma das grandes plataformas de criação de questionários e geração de \textit{leads}. Um dos principais focos da empresa, para além da geração de leads, é a segmentação da audiência. " Interact is a tool for creating online quizzes that generate leads, segment your audience, and drive traffic to your website. "\cite{interact}.



\subsection{easypromo}
\label{easypromoM}

\textit{Easypromos}\cite{f6} é uma plataforma para criação e gestão de campanhas, questionários, concursos, promoções etc.. A \textit{easypromos} afirma que a sua plataforma aumenta o número de seguidores das marcas, melhora a sua visibilidade, gera um maior número de\textit{leads} e ajuda na conversão dos mesmos para clientes.


\subsection{Discussão de funcionalidades}
\label{comparacao}

\subsubsection{Questionários}

Nas tabelas \ref{tab:comparacao1}, \ref{tab:comparacao2} e \ref{tab:comparacao3} encontra-se a comparação entre as plataformas analisadas no Anexo \ref{a:ea}, baseada numa lista de funcionalidades.

Como vimos na análise efetuada no Anexo \ref{a:ea}, em todas as plataformas/ferramentas é necessário criar uma conta para aceder a todas as funcionalidades, contudo apenas o \textit{involve.me}, \textit{easypromos} e o 10.quest fornecem um plano gratuito que apenas dá acesso a algumas funcionalidades. As restantes plataformas apenas disponibilizam um plano trial fornece o acesso a um pacote de funcionalidades pago, durante 6 (\textit{Survey Anyplace}) a 15 (\textit{Interact}) dias. No fim deste prazo, os utilizadores, para continuarem a usufruir das funcionalidades da plataforma terão de subscrever um plano/pacote de funcionalidades.

Todas as ferramentas analisadas disponibilizam uma série de templates excepto o 10.quest que não prevê a integração dessa funcionalidade num futuro próximo. Todas as ferramentas permitem a criação de questionário do zero e todas elas possuem ferramentas de personalização dos questionários.

De entre todas as plataformas analisadas o 10.quest é a unica que permite importar conteúdo previamente feito com recurso a ficheiros externos (i.e. base de dados de perguntas e respostas em CSV). É de notar que as questões importadas para a plataforma através desta funcionalidade, tendo em conta que esse processo é feito através de uma \textit{spreadsheet}, ficheiros de imagens ou vídeo terão de ser adicionados posteriormente.

Cada plataforma tem a seu método de criar um fluxo lógico ou sistema de pontuações para conseguir calcular o resultado segmentado de acordo com as respostas do utilizador final. O \textit{involve.me} desmonstrou ser a ferramenta mais fraca no que a esta funcionalidade diz respeito, visto que apenas se pode associar um resultado possível a uma resposta. O \textit{easypromos} implementa um sistema correlação com as resultados possívo. O utilizador cria uma série de resultados e atribui a cada um número (categoria), o resultado final será o resultado cuja sua categoria foi escolhida mais vezes. Uma desvantagem deste método é que não se consegue atribuir maior importância a uma determinada pergunta. O \textit{Survey Anyplace} e o 10.quest demonstram ser as plataformas com maior capacidade de implementar estas funcionalidades visto que ambos integram um sistema de pontuações e ambos conseguem criar um fluxo lógico. O \textit{Interact} consegue associar resultados a respostas contudo todas as ligações valem o mesmo e por isso mesmo o cálculo do resultado final não é tão preciso.


Depois de terminado um questionário é necessário pré-visualizar o mesmo para verificar se tudo está de acordo com o idealizado e tal como em todas as plataformas analisadas, o 10.quest terá essa funcionalidade.

Para partilhar os questionários, o \textit{Survey Anyplace} fica um pouco atrás de todas as outras plataformas visto que não inclui a partilha dos questionários nas redes sociais, dentro da plataforma. Apesar disso o \textit{Survey Anyplace} gera um link para ser partilhado, tal como em todas as plataformas restantes, e possibilita a integração do questionário noutro \textit{website}. 
%Por decisão do cliente o 10.quest não permite ser embebido noutros \textit{websites}. 
Outra maneira de partilhar os questionários será através de um link ou através das redes sociais. Desta forma é partilhada uma \textit{landing page} onde o utilizador final se inscreve/subscreve e de seguida recebe a formação, questionário ou concurso por email. As únicas plataformas que implementam e automatizam este processo são o \textit{Interact} e o 10.quest.


Na secção de recolha e análise de dados a 10.quest destaca-se. A 10.quest é unica plataforma que apresenta estatísticas globais que satisfazem as necessidades do utilizador (i. e. que representam dados relevantes). É também a única plataforma que tem capacidade de associar as tags dos questionários a utilizadores finais e assim criar perfis de utilizador conseguindo segmentar leads qualificadas. Estas tags são associadas às formações, questionários e concursos, e assim que um uitilizador final se inscreve nas mesmas, a tag é automaticamente associada ao utilizador final.  À semelhaça da plataforma \textit{Interact} o 10.quest mostra o túnel de conversão na análise de dados e gera um gráfico temporal de eventos que ajuda a perceber as tendências e as taxas de conversão ao longo do tempo.
 
 
 \subsubsection{Concursos}
 
 As plataformas \textit{easypromos} e 10.quest são as únicas plataformas capazes de criar concursos. 
 Apesar de todas as plataformas terem múltiplas funcionalidades para criar e personalizar divertos tipos de pergunta, à excepção da \textit{easypromos} e do 10.quest, as restantes plataformas não apresentam os resultados em forma de \textit{leaderboard}, nem fazem o cálculo e atribuição de prémio ao vencedor de forma automática.
 Todas as restantes funcionalidades associadas aos concursos, são muito semelhantes, com as funcionalidades descritas em cima e no Anexo \ref{a:ea}. 
 
 \subsubsection{Tabela de comparações}
	
	
\renewcommand{\arraystretch}{2}
\setlength\arrayrulewidth{1pt}
\begin{table}[!ht]  
	\begin{center}
		\begin{tabular}{|p{3cm}|p{0cm}|p{0cm}|p{0cm}|p{0cm}|p{0cm}|}
			\cline{2-6}
			\multicolumn{1}{c|}{} & \hspace{0.2cm}\begin{sideways}involve.me\end{sideways} & \hspace{0.4cm}\begin{sideways}Survey Anyplace\end{sideways} & \hspace{0.2cm}\begin{sideways}Interact\end{sideways}&
			\hspace{0.2cm}\begin{sideways}easypromos\end{sideways}& \hspace{0.2cm}\begin{sideways} 10.quest\end{sideways}\\ \hline
			
			
			Plano Gratuito & \cellcolor{yellow!80}   & \cellcolor{red!80}  & \cellcolor{red!80} & \cellcolor{yellow!80} & \cellcolor{yellow!80} \\ \hline
			
			Criar questionário novo & \cellcolor{green!80}  & \cellcolor{green!80}  & \cellcolor{green!80} & \cellcolor{green!80}  &\cellcolor{green!80} \\ \hline
			
			Templates de questionários disponíveis& \cellcolor{green!80}  & \cellcolor{green!80} & \cellcolor{green!80} & \cellcolor{green!80}  & \cellcolor{red!80}  \\ \hline		
			
			Importar conteúdo feito previamente & \cellcolor{red!80}   & \cellcolor{red!80}  & \cellcolor{red!80} & \cellcolor{red!80} & \cellcolor{green!80}  \\ \hline
			
			Personalização do questionário & \cellcolor{green!80}  & \cellcolor{green!80}  & \cellcolor{green!80} & \cellcolor{green!80} & \cellcolor{green!80} \\ \hline
			
			Criar perfis de utilizador & \cellcolor{red!80}   & \cellcolor{red!80}  & \cellcolor{red!80} & \cellcolor{red!80} & \cellcolor{green!80}  \\ \hline
			
		\end{tabular}
	\end{center}
	\hspace{1.2cm}	\textcolor{red}{$\blacksquare$} Funcionalidade não implementada
	
	\hspace{1.2cm}     \textcolor{yellow}{$\blacksquare$} Funcionalidade parcialmente implementada (i. e. não satisfaz totalmente as necessidades do utilizador)
	
	\hspace{1.2cm}     \textcolor{green}{$\blacksquare$} Funcionalidade totalmente implementada 
	\begin{center}
		\caption{Tabela de comparação de funcionalidades}
		\label{tab:comparacao1}
	\end{center}
\end{table}

\newpage

\renewcommand{\arraystretch}{2}
\setlength\arrayrulewidth{1pt}
\begin{table}[!ht]  
	\begin{center}
		\begin{tabular}{|p{3cm}|p{0cm}|p{0cm}|p{0cm}|p{0cm}|p{0cm}|}
			\cline{2-6}
			\multicolumn{1}{c|}{} & \hspace{0.2cm}\begin{sideways}involve.me\end{sideways} & \hspace{0.4cm}\begin{sideways}Survey Anyplace\end{sideways} & \hspace{0.2cm}\begin{sideways}Interact\end{sideways}&
			\hspace{0.2cm}\begin{sideways}easypromos\end{sideways}&
			\hspace{0.2cm}\begin{sideways} 10.quest\end{sideways}\\ \hline
		
		
			Pré-visualização dos questionários &\cellcolor{green!80}  & \cellcolor{green!80} & \cellcolor{green!80} & \cellcolor{green!80} & \cellcolor{green!80}  \\ \hline
			
			Exportar os resultados &\cellcolor{green!80}  & \cellcolor{green!80} & \cellcolor{green!80} & \cellcolor{green!80} & \cellcolor{green!80}  \\ \hline
			
			Integração de sistemas externos & \cellcolor{green!80}  & \cellcolor{green!80} & \cellcolor{green!80} & \cellcolor{green!80} & \cellcolor{red!80}  \\ \hline
		
			Partilhar questionários nas redes sociais &\cellcolor{green!80}  & \cellcolor{yellow!80} & \cellcolor{green!80} & \cellcolor{green!80} & \cellcolor{green!80}  \\ \hline
		
			
			Estado dos questionários &\cellcolor{green!80}  & \cellcolor{red!80} & \cellcolor{red!80} & \cellcolor{green!80}  & \cellcolor{green!80}  \\ \hline
			
			Gerador de \textit{Landing Page}  &\cellcolor{red!80}  & \cellcolor{red!80} & \cellcolor{green!80} & \cellcolor{red!80}  & \cellcolor{green!80}  \\ \hline
			
			Email Marketing &\cellcolor{red!80}  & \cellcolor{green!80} & \cellcolor{green!80} & \cellcolor{green!80} & \cellcolor{green!80}  \\ \hline
			
			 Embeber questionários& \cellcolor{green!80}  & \cellcolor{green!80}  & \cellcolor{green!80} & \cellcolor{green!80} & \cellcolor{green!80}  \\ \hline
			
			Fluxo Lógico/Sistema de pontuação &\cellcolor{yellow!80}  & \cellcolor{green!80} & \cellcolor{green!80} & \cellcolor{yellow!80} & \cellcolor{green!80}  \\ \hline
			
		
			
		\end{tabular}
	\end{center}
	\hspace{1.2cm}	\textcolor{red}{$\blacksquare$} Funcionalidade não implementada
	
	\hspace{1.2cm}     \textcolor{yellow}{$\blacksquare$} Funcionalidade parcialmente implementada
	
	\hspace{1.2cm}     \textcolor{green}{$\blacksquare$} Funcionalidade totalmente implementada 
	\begin{center}
		\caption{Tabela de comparação de funcionalidades (continuação)}
		\label{tab:comparacao2}
	\end{center}
\end{table}


\newpage

\renewcommand{\arraystretch}{2}
\setlength\arrayrulewidth{1pt}
\begin{table}[!ht]  
	\begin{center}
		\begin{tabular}{|p{3cm}|p{0cm}|p{0cm}|p{0cm}|p{0cm}|p{0cm}|}
			\cline{2-6}
			\multicolumn{1}{c|}{} & \hspace{0.2cm}\begin{sideways}involve.me\end{sideways} & \hspace{0.4cm}\begin{sideways}Survey Anyplace\end{sideways} & \hspace{0.2cm}\begin{sideways}Interact\end{sideways}&
			\hspace{0.2cm}\begin{sideways}easypromos\end{sideways}&
			\hspace{0.2cm}\begin{sideways} 10.quest\end{sideways}\\ \hline
			
			Design responsivo & \cellcolor{green!80}  & \cellcolor{green!80} & \cellcolor{green!80} & \cellcolor{green!80}  & \cellcolor{green!80}  \\ \hline			
			
			Análise e segmentação de resultados & \cellcolor{green!80}  & \cellcolor{green!80}  & \cellcolor{green!80} & \cellcolor{green!80} & \cellcolor{green!80}  \\ \hline
			
			Estatísticas gerais & \cellcolor{yellow!80}  & \cellcolor{red!80}  & \cellcolor{red!80} & \cellcolor{red!80} & \cellcolor{green!80} \\ \hline
			
			Criar perfis de utilizador & \cellcolor{red!80}  & \cellcolor{red!80}  & \cellcolor{red!80} & \cellcolor{red!80} & \cellcolor{green!80}  \\ \hline
			
			Suporte dedicado (i. e. canal de comunicação dedicado para suporte da plataforma)		 & \cellcolor{green!80}  & \cellcolor{green!80}  & \cellcolor{green!80} & \cellcolor{green!80} & \cellcolor{green!80}  \\ \hline
			
		\end{tabular}
		\end{center}
	\hspace{1.2cm}	\textcolor{red}{$\blacksquare$} Funcionalidade não implementada

\hspace{1.2cm}     \textcolor{yellow}{$\blacksquare$} Funcionalidade parcialmente implementada

\hspace{1.2cm}     \textcolor{green}{$\blacksquare$} Funcionalidade totalmente implementada 
\begin{center}
\caption{Tabela de comparação de funcionalidades (continuação)}
\label{tab:comparacao3}
\end{center}
\end{table}

\pagebreak
%-------------------------------------------------------------------------------------------------
\blankpage
%-------------------------------------------------------------------------------------------------

\glsresetall



