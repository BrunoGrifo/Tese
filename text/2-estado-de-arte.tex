\chapter{Estado de Arte}
\label{sec:estado-arte}

Neste capítulo será feita uma análise de algumas plataformas existentes para recolha de dados em estratégias de inbound marketing. A recolha de informação nos dias de hoje tem um grande impacto na forma como os negócios são feitos, principalmente na internet. Neste contexto, serão analisadas algumas das soluções que partilham funcionalidades semelhantes com o que vai ser desenvolvido.\\
Como referido no capítulo anterior, o modelo de negócio da plataforma é \gls{b2b} e neste sentido as caracteristicas analisadas serão focadas na segmentação de dados, integração com serviços externos, planos de pagamento e funcionalidades do \textit{back office}.
Seguindo uma das estratégias de inbound marketing, o método de envio de formações e recolha de dados será através da formulários online e por isso mesmo algumas características associadas à experiencia do utilizador, como por exemplo a personalização do formulário, serão também analisadas.


As plataformas a analisar serão: SurveyMonkey\cite{surveymonkey}, Typeform\cite{typeform}, Google Forms\cite{googleform}

Após a apresentação destas ferramentas será feita uma análise das vantagens e desvantagens de cada uma, assim como a comparação de funcionalidades.


\section{SurveyMonkey}
\label{surveyMonkey}

O SurveyMonkey é uma plataforma \acrfull{saas} de criação de formulários online. É necessário criar conta para aceder às funcionalidades da plataforma, dando a opção de utilizar serviços externos para esse efeito (e. g. Facebook\cite{face}, LinkedIn\cite{linkedin}), como podemos ver na Figura \ref{fig:surveymonkey-singin}. O SurveyMonkey é uma plataforma que dispões de diversos planos de pagamento, e por isso mesmo, apesar de estar disponível um plano gratuito, tem acesso apenas a algumas das funcionalidades e em cada formulário, no máximo, poderá ter 10 perguntas ou elementos. 

\begin{figure}[ht!]
	\begin{center}
		\includegraphics[width=1\textwidth]{img/surveymonkey-singin}
		\caption{SurveyMonkey - Registro }
		\label{fig:surveymonkey-singin}
	\end{center}
\end{figure}

\newpage

No painel principal, como podemos ver na Figura \ref{fig:survey-dashboard} temos acesso rápido aos formulários recentes e a algumas métricas sobre os mesmos. Outra forma será aceder aos formulários do utilizador através da barra de navegação. 


\begin{figure}[ht!]
	\begin{center}
		\includegraphics[width=1\textwidth]{img/survey-dashboard}
		\caption{SurveyMonkey - Painel de Controle }
		\label{fig:survey-dashboard}
	\end{center}
\end{figure}

\newpage

\begin{figure}[ht!]
	\begin{center}
		\includegraphics[width=1\textwidth]{img/survey-form-create}
		\caption{SurveyMonkey - Formulários modelo }
		\label{fig:survey-form-create}
	\end{center}
\end{figure}

Quando se inicializa a criação de um novo formulário, a plataforma dá opção de começar do zero ou de utilizar um formulário modelo como podemos ver na Figura \ref{fig:survey-form-create}. Começando um formulário do zero como podemos ver na Figura \ref{fig:survey-form-banck2}, temos acesso a uma série de funcionalidades que vamos explorar e analisar em seguida.

\begin{figure}[ht!]
	\begin{center}
		\includegraphics[width=1\textwidth]{img/survey-form-bank2}
		\caption{SurveyMonkey -  Perguntas Modelo}
		\label{fig:survey-form-banck2}
	\end{center}
\end{figure}


\begin{figure}[ht!]
	\begin{center}
		\includegraphics[width=1\textwidth]{img/survey-form-bank1}
		\caption{SurveyMonkey - Perguntas Modelo }
		\label{fig:survey-form-banck1}
	\end{center}
\end{figure}

\newpage

São diversos os elementos que se podem adicionar ou arrastar para o formulário (i. e. perguntas, escolha multipla, imagens...) como representado na Figura \ref{fig:surveymonkey-form-element} e há também um banco de perguntas modelo/recomendações já construídas, organizadas por categorias como podemos ver na Figura \ref{fig:survey-form-banck2} e \ref{fig:survey-form-banck1}.


\begin{figure}[ht!]
	\begin{center}
		\includegraphics[width=1\textwidth]{img/surveymonkey-form-element}
		\caption{SurveyMonkey - Elementos }
		\label{fig:surveymonkey-form-element}
	\end{center}
\end{figure}
\newpage

\begin{figure}[ht!]
	\begin{center}
		\includegraphics[height=.32\textheight]{img/surveymonkey-form-opcoes}
		\caption{SurveyMonkey - Opções}
		\label{fig:surveymonkey-form-opcoes}
	\end{center}
\end{figure}

\begin{figure}[ht!]
	\begin{center}
		\begin{minipage}{0.45\textwidth}
			\begin{center}
				\includegraphics[height=.32\textheight]{img/surveymonkey-form-aparencia}
				\caption{SurveyMonkey - Aparência}
				\label{fig:surveymonkey-form-aparencia}
			\end{center}
		\end{minipage}
		\hspace{1cm}
		\begin{minipage}{0.45\textwidth}
			\begin{center}
				\includegraphics[height=.32\textheight]{img/surveymonkey-form-logica}
				\caption{SurveyMonkey - Lógica}
				\label{fig:surveymonkey-form-logica}
			\end{center}
		\end{minipage}
	\end{center}
\end{figure}

O SurveyMonkey permite também realizar algumas operações de personalização do formulário.Na Figuras \ref{fig:surveymonkey-form-opcoes}, \ref{fig:surveymonkey-form-aparencia} e \ref{fig:surveymonkey-form-logica} estão representaçãs as opções, aparência e lógica do formulário, respetivamente, que permite costumizar formulários ao público alvo. 
Depois de realizado o formulário esta plataforma permite a visualização do mesmo, em diferentes tipos de dispositivos, como se pode ver nas Figuras \ref{fig:surveymonkey-form-test-pc} e \ref{fig:surveymonkey-form-test-phone}, para verificar se tudo está conforme planeado para se poder prosseguir para a recolha de dados. 
 \newpage

\begin{figure}[ht!]
	\begin{center}
		\includegraphics[width=1\textwidth]{img/surveymonkey-form-test-pc}
		\caption{SurveyMonkey - Visualização do formulário em computador }
		\label{fig:surveymonkey-form-test-pc}
	\end{center}
\end{figure}

\begin{figure}[ht!]
	\begin{center}
		\includegraphics[width=1\textwidth]{img/surveymonkey-form-test-phone}
		\caption{SurveyMonkey - Visualisação do formulário em smartphone }
		\label{fig:surveymonkey-form-test-phone}
	\end{center}
\end{figure}

Depois de garantir que o formulário foi construido como desejado a plataforma fornece vários meios pelo qual se pode partilhar/enviar o formulário, como listado na Figura \ref{fig:surveymonkey-form-share}.

\newpage

\begin{figure}[ht!]
	\begin{center}
		\includegraphics[width=1\textwidth]{img/surveymonkey-form-share}
		\caption{SurveyMonkey - Metodo de partilha do formulário }
		\label{fig:surveymonkey-form-share}
	\end{center}
\end{figure}

Na analise de resultados, é necessário actualizar a página ou aplicar um filtro para que os gráficos e as estatísticas sejam actualizadas. Para aplicar um filtro é necessário escolher o tipo de filtro e os elementos ao qual queres aplicar o filtro como podemos ver nas Figuras \ref{fig:surveymonkey-form-filtro} e \ref{fig:surveymonkey-form-filtro1}.


\begin{figure}[ht!]
	\begin{center}
		\includegraphics[width=1\textwidth]{img/surveymonkey-form-filtro}
		\caption{SurveyMonkey - Tipos de Filtros }
		\label{fig:surveymonkey-form-filtro}
	\end{center}
\end{figure}

\begin{figure}[ht!]
\begin{center}
	\includegraphics[width=1\textwidth]{img/surveymonkey-form-filtro1}
	\caption{SurveyMonkey - Filtro aplicado na pergunta 2 }
	\label{fig:surveymonkey-form-filtro1}
\end{center}
\end{figure}

\section{Typeform}
\label{typeform}

O Typeform é uma plataforma \acrshort{saas} de criação de formulários online.



Typeform Help Center. Typeform makes collecting and sharing information comfortable and conversational. It's a web based platform you can use to create anything from surveys to apps, without needing to write a single line of code. Online forms are boring – typeforms fix that

Typeform is a Barcelona-based online software as a service (SaaS) company that specializes in online form building and online surveys. Its main software creates dynamic forms based on user needs.


\section{Google Form}
\label{googleform}
%-------------------------------------------------------------------------------------------------
\blankpage
%-------------------------------------------------------------------------------------------------

\glsresetall



