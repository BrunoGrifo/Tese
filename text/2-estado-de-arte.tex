\chapter{Estado de Arte}
\label{sec:estado-arte}

Neste capítulo será feita a comparação de algumas plataformas existentes para recolha de dados em estratégias de inbound marketing. Será também feita um resumo das funcionalidades do \acrshort{tcg}, que apesar de já estar no mercado, será integrado neste projeto. Tendo em conta que a plataforma a desenvolver irá utilizar as funcionalidades para criação de formações do \acrshort{tcg} e que a equipa do \acrlong{tcg} já fez o estudo de mercado antes do desenvolvimento do mesmo, e continua a fazer todos os dias, a análise de plataformas focadas em criação de formações não será feita neste documento. Uma análise mais detalhadas de todas as plataformas estudadas encontra-se no Anexo \ref{a:ea}.

A recolha de informação nos dias de hoje tem um grande impacto na forma como os negócios são feitos, principalmente na internet.

Como referido no capítulo anterior, o objectivo deste estágio incide na criação de uma plataforma de inbound marketing que tem como principais objectivos conseguir criar concursos, questionários, formações e analisar, filtrar e segmentar os dados recolhidos. Neste sentido serão analisadas algumas das soluções existentes que, apesar de algumas terem um propósito distinto, podem ser utilizadas em estratégias de inbound marketing e partilham funcionalidades semelhantes com o que vai ser desenvolvido. 

A recolha de informação nos dias de hoje tem um grande impacto na forma como os negócios são feitos, principalmente na internet. Seguindo uma das estratégias de inbound marketing, o método de recolha de dados será através de concursos, questionários e formações online e por isso mesmo algumas características associadas à experiência do utilizador, como por exemplo a personalização dos mesmos, serão também analisadas.

%Rever o que está para baixo

As plataformas SurveyMonkey\cite{surveymonkey}, Typeform\cite{typeform}, Google Forms\cite{googleform} são plataformas de criação de formulários, mas apesar de servirem um propósito distinto ao da plataforma a desenvolver, partilham funcionalidades semelhantes com as que vão ser desenvolvidas e por isso mesmo senão analisadas nesse sentido. Será também exposto o funcionamento do \acrshort{tcg} para melhor entendimento de como podem ser implementadas as funcionalidades e a \acrshort{api} do lado da plataforma a desenvolver. A integração do \acrshort{tcg} neste projeto, foi uma escolha da empresa (i. e. cliente), e por isso não serão analisadas plataformas concorrentes do \acrshort{tcg} na medida em que esse trabalho é feito pela equipa do \acrshort{tcg} e apenas serão feitas as mudanças necessárias no mesmo para o desenvolvimento da \acrshort{api} de comunicação. Por último serão analisadas as plataformas involve.me, Survey Anyplace e Interact que são algumas das serviços que podem concorrer directamente com o 10.quest (i. e. plataforma a desenvolver).

Após a apresentação destas ferramentas será feita uma análise das vantagens e desvantagens de cada uma, assim como a comparação de funcionalidades.

\section{Plataformas de criação de formulários}
\label{formulários}

\subsection{SurveyMonkey}
\label{surveyMonkeyM}

O SurveyMonkey é uma plataforma \acrfull{saas} de criação de formulários online para estudo de mercado, análise de competidores, feedback de clientes e colaboradores, entre outros. Permite recolher informações do público alvo através de formulários e personalizar, segmentar e visualizar esta informação.

As várias funcionalidades do SurveyMonkey foram desenhadas para ajudar os utilizadores a criar diferentes tipos de formulários e recolher os resultados em tempo real. Estes dados são depois analisados e reportados através de funcionalidades próprias do SurveyMonkey. Estes dados podem ainda ser exportados para satisfazer outras necessidades. 

\subsection{Typeform}
\label{typeformM}

O Typeform é uma plataforma \acrshort{saas} de criação de formulários online. É uma empresa que afirma resolver o problema dos formulários e inquéritos aborrecidos e tem também como proposta de valor o facto de conseguir criar formulários e inquéritos sem ter que programar uma única linha de código. Esta plataforma permite recolher informações do público alvo através de formulários e inquéritos personalizados, podendo no final visualizar estes dados. 


\subsection{Google Form}
\label{googleformM}

O Google Form é uma aplicação de administração de inquéritos que está incluída no Google Drive office juntamente com o Google Docs\cite{gdocs}, Google Sheets e Google Slides\cite{gslides}. Esta ferramenta permite recolher informações do público alvo através de formulários e inquéritos personalizados e automaticamente exportar os dados para uma \textit{google sheet}.

\subsection{Análise de concorrentes indirectos}

Nesta secção serão analisadas e comparadas três das ferramentas líderes na criação de formulários online, listadas anteriormente. Como foi referido anteriormente, apesar do SurveyMonkey, Typeform e Google Form serem plataformas que não concorrem com 10.quest e têm um propósito distinto, partilham funcionalidades semelhantes com a plataforma que vai ser desenvolvida. Dito isto é importante perceber quais são estas funcionalidades e tentar perceber como as podemos aproveitar ou melhorar.

Como podemos observar na análise da secção \ref{a:ea} em todas as plataformas/ferramentas é necessário criar uma conta para aceder a todas as funcionalidades e em todas as plataformas analisadas é possível criar conta e iniciar sessão através de sistemas externos (\acrshort{api}). O Projeto a desenvolver segue um modelo \gls{b2b} e por isso mesmo não é de grande importância implementar esse tipo de funcionalidades.

O plataforma da Google fornece, no plano gratuito, todas as funcionalidades, ao contrário do SurveyMonkey e do Typeform, que, tal como se passará com a plataforma a desenvolver, para se ter acesso a todas as funcionalidades, ou pacotes de funcionalidades, terá de ser paga uma subscrição.

Todas as plataformas permitem a criação de formulários do zero, e a plataforma da 10.digital não é excepção. Tal como foi definido na estratégia de negócio, os conteúdos que serão lançados nas formações, questionários e concursos não serão da autoridade da 10.digital, a menos que estejamos incluídos em algum projeto relacionado. Dito isto facilmente se decidiu que a plataforma a desenvolver terá, tal como todas as outras ferramentas analisadas, a funcionalidade de poder adicionar conteúdo previamente feito de forma rápido (i. e. adicionar um ficheiro com todo o conteúdo estruturado). 


Tal como foi referido no capítulo \ref{sec:introducao}, secção \ref{subsec:contexto}, o inbound é uma estratégia de marketing que se foca em criar razões para os público alvo vir até nós através da criação de conteúdo interessante, útil, relevante etc... Para manter esta procura por parte dos clientes é necessário haver valor ao longo da jornada e idealmente proporcionar uma boa experiência ao utilizador. Nesta medida a personalização das formações, questionários e concursos é muito importante tanto a nível de conteúdo como estético e funcional para que o utilizador se sinta valorizado. Para tornar isto possível, tal como o SurveyMonkey, Typeform e Google Form, a plataforma a desenvolver terá algumas funcionalidades que lhe permitirá personalizar as perguntas para melhorar a experiência do utilizador final. 

Como pudemos ver na análise da secção \ref{a:ea} o SurveyMonkey e o Typeform implementam a funcionalidade da criação de um fluxo lógico, contudo esta funcionalidade será analisada mais à frente visto que será mais complexa e comparada com as plataformas directamente concorrentes.

Antes de enviar/partilhar um formulário é sempre importante pré-visualizar e testar. Neste aspecto a plataforma a desenvolver não é diferente. Será possível pré-visualizar as formações, questionários e concursos para verificar e validar os mesmos. 

Por fim temos a análise e tratamento de dados que é um dos suportes daquilo que é o tripé do marketing digital. Como será de esperar a plataforma a desenvolver, tal como todas as restantes plataformas analisadas, não é um software dedicado à análise e tratamento de dados, na medida que terá limitações, contudo, terá as funcionalidades necessárias para satisfazer as necessidades do utilizador. A plataforma SurveyMonkey, neste aspecto, foi a única ferramenta que apresentou funcionalidades para além da análise básica conseguindo, de forma tímida, segmentar e comparar resultados, ao contrário do Typeform e do Google Form que apenas apresentam os resultados globais e por pergunta.

\section{\acrfull{tcg}}
\label{sec:TCGM}

O \acrlong{tcg} é um produto actualmente no mercado, desenvolvido pela equipa da 10.digital, que tem como principal objectivo transformar PDFs numa aprendizagem baseada em tentativa erro.

 O \acrshort{tcg} nasceu de uma forte convicção de que perder apenas 2 minutos por dia numa formação tentativa erro é uma óptima forma de aprender, poupando tempo e dinheiro às empresas. Inicialmente muito focado em formação interna, a equipa do \acrshort{tcg} foi-se apercebendo que existem muitos outros problemas (e. g. Consolidação de procedimentos, \textit{Onboarding} de novos colaboradores, Divulgação da cultura da empresa, Divulgação de informações técnicas a parceiros/clientes ...) para o qual a plataforma tem solução (e. g. Assimilação da cultura de empresa e do espírito das marcas, Simplificação do processo de acolhimento, Redução de custos em reuniões periódicas, Facilidade em divulgar aspectos técnicos, que de outra, forma demorariam mais tempo ...).\cite{tcginfo}. 
 
 As principais funcionalidades do  \acrshort{tcg}  são as seguintes:


\begin{itemize}
	\item[--] Criação de formações 
	\item[--] Tutoriais para as funcionalidades chave
	\item[--] Algoritmo de gestão automática (i. e. sistema identifica as perguntas mais difíceis baseado nos resultados do utilizador final. As perguntas mais difíceis saem com mais frequência e perguntas erradas saem no dia seguinte)
	\item[--] Feedback dos utilizadores em cada pergunta
	\item[--] \textit{Gamification}
	\item[--] Análise detalhada dos resultados
	\item[--] Compatível com dispositivos móveis
\end{itemize}

\section{Análise de concorrentes directos}

Um dos objectivos do projecto é possibilitar aos utilizadores da plataforma conseguirem criar questionários que, baseado nas respostas do utilizador final, apresente um resultado no fim do mesmo, tal como referido no Capítulo \ref{subsec:objetivos}.

O Akinator\cite{akinator} e o high5test\cite{5} são dois bons exemplos de aplicações que através de um questionário, e baseado nas respostas do utilizador final, apresenta um resultado. Estas duas aplicações são bastante poderosas e utilizam algoritmos de decisão que os permitem tirar este tipo de conclusões. Nesta medida alguns algoritmos como o CART\cite{cart}, ID3\cite{id3}\cite{id3_2}\cite{cart} e C4.5\cite{cart}\cite{c4.5} foram brevemente estudados com intuito de serem aplicados nesta funcionalidade do projecto. Este tipo de algoritmos, não só são complexos como são algoritmos que necessitam de dados de treino. Tal como referido no Capítulo \ref{subsec:objetivos}, um dos requisitos definidos pelo cliente é que a criação destes questionário seja intuitiva e exequível por qualquer pessoa mesmo que esta não tenha qualquer conhecimento em programação. Digo isto e somando o facto de que, em qualquer momento da criação dos questionário, não haverá dados de treino, a aplicação deste tipo de algoritmos não será possível.

Dito isto foi pensado outras abordagens e determinou-se que um resultado semelhante (i. e. satisfazendo as necessidades do cliente) consegue ser alcançado com um sistema de pontuações. Nesta abordagem são atribuídos pesos a cada pergunta e consoante a resposta do utilizador, o sistema vai pontuando os resultados possíveis para que no final possa ser apresentado o resultado com maior pontuação.

Uma aplicação deste tipo de questionário será, por exemplo: "Calcule o lugar ideal de para viajar nas suas férias". Actualmente, num formato semelhante (i. e. em formato \textit{quiz}) já existem alguns \textit{Websites}(e. g. Chase for Adventure\cite{chaseforadventure}, travelpicker\cite{travelpicker}, Insight Vacations\cite{insightvacations} e Driftwood Journals\cite{driftwoodjournals}) que satisfazem, de forma simplista, esta necessidade em específico. Apesar de simples são exemplos que se conseguiriam replicar com a plataforma a desenvolver e que satisfazem as necessidades do cliente. Neste sentido foram analisadas plataformas mais poderosas, que se focam na construção de leads, e que conseguem o mesmo resultado.


\subsection{involve.me}
\label{involvemeM}

%Não aceita multiplos resultados por resposta
%não cria perfis de utilizador
%multichannel supporte
%landiung page

"\textit{involve.me is a next-generation user engagement \& customer experience platform with a focus on digital marketers \& e-commerce.}"\cite{involve}. O involve-me é uma plataforma moderna que ajuda empresas a criar interações personalizadas ao longo da jornada dos clientes, aumentando a audiência e recolhendo mais e melhores dados. Esta plataforma foca-se também na recolha e análise destes dados/informações sobre os utilizadores finais.


\subsection{Survey Anyplace}
\label{surveyanyplaceM}


%nao avisa que está live o questionário
%feature de ajuda no canto inferior direito -  multi channel


O Survey Anyplace é uma plataforma online com foco na criação de \textit{surveys} e questionários interativos. O Survey Anyplace afirma proporcionar uma boa experiência para o utilizador através de elementos interativos e funcionalidades de personalização. Esta plataforma permite também a análise dos dados recolhidos através dos questionários publicados.


\subsection{Interact}
\label{interactM}


O Interact é uma das grandes plataformas de criação de questionários e geração de leads. Um dos grandes focos da empresa é a geração de leads e segmentação da audiência. " Interact is a tool for creating online quizzes that generate leads, segment your audience, and drive traffic to your website. "\cite{interact}.


\subsection{Discussão de funcionalidades}
\label{comparacao}

Nas tabelas \ref{tab:comparacao1}, \ref{tab:comparacao2} e \ref{tab:comparacao3} encontra-se a comparação entre as plataformas analisadas no Anexo \ref{a:ea}, baseada numa lista de funcionalidades.

Como vimos na análise efetuada no Anexo \ref{a:ea}, em todas as plataformas/ferramentas é necessário criar uma conta para aceder a todas as funcionalidades, contudo apenas o involve.me e o 10.quest fornecem um plano gratuito que apenas dá acesso a algumas funcionalidades. As restantes plataformas apenas disponibilizam um plano trail de 6 e 15 dias e para aceder a todas as funcionalidades, tal como no involve.me e no 10.quest é necessário subscrever um plano pago.

Todas as ferramentas analisadas disponibilizam uma série de templates excepto o 10.quest que, para um futuro próximo não terá essa funcionalidade. Todas as ferramentas permitem a criação de questionário do zero e todas elas possuem ferramentas de personalização dos questionários.

De entre todas as plataformas analisadas o 10.quest é a unica que permite importar conteúdo previamente feito. É de notar que as questões importadas para a plataforma através desta funcionalidade, tendo em conta que esse processo é feito através de uma spreadsheet, ficheiros de imagens ou vídeo terão de ser adicionados posteriormente.

Cada plataforma tem a seu método de criar um fluxo lógico ou sistema de pontuações para conseguir calcular o resultado final de acordo com as respostas do utilizador final. O involve.me desmonstrou ser a ferramenta mais fraca visto que apenas se pode associar um resultado possível a uma resposta. O Survey Anyplace e o 10.quest demonstram ser as plataformas com maior capacidade de implementar estas funcionalidades visto que ambos implementa um sistema de pontuações e ambos conseguem criar um fluxo lógico. O Interact consegue associar resultados a respostas contudo todas as ligações valem o mesmo e por isso mesmo o cálculo do resultado final não é tão preciso. 

Depois de terminado um questionário é necessário pré-visualizar o mesmo para verificar se tudo ficou como desejado e tal como em todas as plataformas analisadas, o 10.quest terá essa funcionalidade. 

Para partilhar os questionários, o Survey Anyplace fica um pouco atrás de todas as outras plataformas visto que não implementa a partilha dos questionários nas redes sociais, dentro da plataforma. Apesar disso o Survey Anyplace gera um link para ser partilhado, tal como em todas as plataformas restantes, e possibilita a integração do questionário noutro website. Por decisão do cliente o 10.quest não permite ser embebido noutros websites.  Outras maneiras de partilhar os questionários será através da inscrição em landing pages e os questionários serão automaticamente enviados para o contacto introduzido na inscrição da landing page. As únicas plataformas que implementam e automatizam este processo são o Interact e o 10.quest. 

Na secção de recolha e análise de dados a 10.quest destaca-se. A 10.quest é unica plataforma que apresenta estatísticas globais que satisfazem as necessidades do utilizador. É também a única plataforma que tem capacidade de associar as tags dos questionários a utilizadores finais e assim criar perfis de utilizador conseguindo segmentar leads qualificadas. À semelhaça da plataforma Interact o 10.quest mostra o túnel de conversão na análise de dados e gera um gráfico temporal de eventos que ajuda a perceber as tendências e as taxas de conversão ao longo do tempo.
 


	
	
\renewcommand{\arraystretch}{2}
\setlength\arrayrulewidth{1pt}
\begin{table}[!ht]  
	\begin{center}
		\begin{tabular}{|p{4cm}|p{0.1cm}|p{0.1cm}|p{0.1cm}|p{0.1cm}|}
			\cline{2-5}
			\multicolumn{1}{c|}{} & \hspace{0.2cm}\begin{sideways}involve.me\end{sideways} & \hspace{0.4cm}\begin{sideways}Survey Anyplace\end{sideways} & \hspace{0.2cm}\begin{sideways}Interact\end{sideways} &\hspace{0.2cm}\begin{sideways} 10.quest\end{sideways}\\ \hline
			
			
			Plano Gratuito & \cellcolor{yellow!80}   & \cellcolor{red!80}  & \cellcolor{red!80} & \cellcolor{yellow!80}  \\ \hline
			
			Criar questionário novo & \cellcolor{green!80}  & \cellcolor{green!80}  & \cellcolor{green!80} & \cellcolor{green!80} \\ \hline
			
			Templates de questionários disponíveis& \cellcolor{green!80}  & \cellcolor{green!80} & \cellcolor{green!80} & \cellcolor{red!80}  \\ \hline		
			
			Adicionar conteúdo previamente feitos & \cellcolor{red!80}   & \cellcolor{red!80}  & \cellcolor{red!80} & \cellcolor{green!80}  \\ \hline
			
			
			
		\end{tabular}
	\end{center}
	\hspace{1.2cm}	\textcolor{red}{$\blacksquare$} Funcionalidade não implementada
	
	\hspace{1.2cm}     \textcolor{yellow}{$\blacksquare$} Funcionalidade parcialmente implementada (i. e. não satisfaz totalmente as necessidades do utilizador)
	
	\hspace{1.2cm}     \textcolor{green}{$\blacksquare$} Funcionalidade totalmente implementada 
	\begin{center}
		\caption{Tabela de comparação de funcionalidades}
		\label{tab:comparacao1}
	\end{center}
\end{table}

\newpage

\renewcommand{\arraystretch}{2}
\setlength\arrayrulewidth{1pt}
\begin{table}[!ht]  
	\begin{center}
		\begin{tabular}{|p{4cm}|p{0.1cm}|p{0.1cm}|p{0.1cm}|p{0.1cm}|}
			\cline{2-5}
			\multicolumn{1}{c|}{} & \hspace{0.2cm}\begin{sideways}involve.me\end{sideways} & \hspace{0.4cm}\begin{sideways}Survey Anyplace\end{sideways} & \hspace{0.2cm}\begin{sideways}Interact\end{sideways} &\hspace{0.2cm}\begin{sideways} 10.quest\end{sideways}\\ \hline
			
			
			Personalização do questionário & \cellcolor{green!80}  & \cellcolor{green!80}  & \cellcolor{green!80} & \cellcolor{green!80} \\ \hline
			
			
			Criar perfis de utilizador & \cellcolor{red!80}   & \cellcolor{red!80}  & \cellcolor{red!80} & \cellcolor{green!80}  \\ \hline
					
			Pré-visualização dos questionários &\cellcolor{green!80}  & \cellcolor{green!80} & \cellcolor{green!80} & \cellcolor{green!80}  \\ \hline
			
			Exportar os resultados &\cellcolor{green!80}  & \cellcolor{green!80} & \cellcolor{green!80} & \cellcolor{green!80}  \\ \hline
			
			Integração de sistemas externos & \cellcolor{green!80}  & \cellcolor{green!80} & \cellcolor{green!80} & \cellcolor{red!80}  \\ \hline
			
			Partilhar questionários nas redes sociais &\cellcolor{green!80}  & \cellcolor{yellow!80} & \cellcolor{green!80} & \cellcolor{green!80}  \\ \hline
			
			Estado dos questionários &\cellcolor{green!80}  & \cellcolor{red!80} & \cellcolor{red!80} & \cellcolor{green!80}  \\ \hline
			
			\textit{Landing Page}  &\cellcolor{red!80}  & \cellcolor{red!80} & \cellcolor{green!80} & \cellcolor{green!80}  \\ \hline
			
			Email Marketing &\cellcolor{red!80}  & \cellcolor{green!80} & \cellcolor{green!80} & \cellcolor{green!80}  \\ \hline
			
			Fluxo Lógico/Sistema de pontuação &\cellcolor{yellow!80}  & \cellcolor{green!80} & \cellcolor{green!80} & \cellcolor{green!80}  \\ \hline
			
			Design responsivo para múltiplos dispositivos & \cellcolor{green!80}  & \cellcolor{green!80} & \cellcolor{green!80} & \cellcolor{green!80}  \\ \hline			
			
		\end{tabular}
	\end{center}
	\hspace{1.2cm}	\textcolor{red}{$\blacksquare$} Funcionalidade não implementada
	
	\hspace{1.2cm}     \textcolor{yellow}{$\blacksquare$} Funcionalidade parcialmente implementada
	
	\hspace{1.2cm}     \textcolor{green}{$\blacksquare$} Funcionalidade totalmente implementada 
	\begin{center}
		\caption{Tabela de comparação de funcionalidades (continuação)}
		\label{tab:comparacao2}
	\end{center}
\end{table}



\renewcommand{\arraystretch}{2}
\setlength\arrayrulewidth{1pt}
\begin{table}[!ht]  
	\begin{center}
		\begin{tabular}{|p{4cm}|p{0.1cm}|p{0.1cm}|p{0.1cm}|p{0.1cm}|}
			\cline{2-5}
			\multicolumn{1}{c|}{} & \hspace{0.2cm}\begin{sideways}involve.me\end{sideways} & \hspace{0.4cm}\begin{sideways}Survey Anyplace\end{sideways} & \hspace{0.2cm}\begin{sideways}Interact\end{sideways} &\hspace{0.2cm}\begin{sideways} 10.quest\end{sideways}\\ \hline
			
			Análise e segmentação de resultados & \cellcolor{green!80}  & \cellcolor{green!80}  & \cellcolor{green!80} & \cellcolor{green!80} \\ \hline
			
			Estatísticas gerais & \cellcolor{yellow!80}  & \cellcolor{red!80}  & \cellcolor{red!80} & \cellcolor{green!80} \\ \hline
			
			Suporte dedicado & \cellcolor{green!80}  & \cellcolor{green!80}  & \cellcolor{green!80} & \cellcolor{green!80} \\ \hline
			
			
			
			
		\end{tabular}
	\end{center}
	\hspace{1.2cm}	\textcolor{red}{$\blacksquare$} Funcionalidade não implementada
	
	\hspace{1.2cm}     \textcolor{yellow}{$\blacksquare$} Funcionalidade parcialmente implementada
	
	\hspace{1.2cm}     \textcolor{green}{$\blacksquare$} Funcionalidade totalmente implementada 
	\begin{center}
		\caption{Tabela de comparação de funcionalidades (continuação)}
		\label{tab:comparacao3}
	\end{center}
\end{table}

\pagebreak
%-------------------------------------------------------------------------------------------------
\blankpage
%-------------------------------------------------------------------------------------------------

\glsresetall



