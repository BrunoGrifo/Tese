\chapter{Estado de Arte}
\label{sec:estado-arte}

Neste capítulo será feita a análise de algumas plataformas existentes para recolha de dados em estratégias de inbound marketing. Será também feita uma analise ao \acrshort{tcg}, que apesar de já estar no mercado, será integrado neste projeto. Tendo em conta que a plataforma a desenvolver irá utilizar as funcionalidades para criação de formações do \acrshort{tcg} e que a equipa do \acrlong{tcg} já fez o estudo de mercado antes do desenvolvimento do mesmo, e continua a fazer todos os dias, a analise de plataformas focadas em criação de formaçoes não será feita neste documento. 

A recolha de informação nos dias de hoje tem um grande impacto na forma como os negócios são feitos, principalmente na internet.

Como referido no capítulo anterior, o objectivo deste estágio incide na criação de uma plataforma de inbound marketing que tem como principais objectivos conseguir criar concursos, questionários, formações e analisar, filtrar e segmentar os dados recolhidos. Neste sentido serão analisadas algumas das soluções existentes que, apesar algumas terem um propósito distinto, podem ser utilizadas em estratégias de inbound marketing e partilham funcionalidades semelhantes com o que vai ser desenvolvido. 

A recolha de informação nos dias de hoje tem um grande impacto na forma como os negócios são feitos, principalmente na internet. Seguindo uma das estratégias de inbound marketing, o método de recolha de dados será através de concursos, questionários e formações online e por isso mesmo algumas características associadas à experiência do utilizador, como por exemplo a personalização dos mesmos, serão também analisadas.

%Rever o que está para baixo

As plataformas SurveyMonkey\cite{surveymonkey}, Typeform\cite{typeform}, Google Forms\cite{googleform} são plataformas de criação de formulários, mas apesar de servirem um propósito distinto ao da plataforma a desenvolver, partilham funcionalidades semelhantes com as que vão ser desenvolvidas e por isso mesmo senão analisadas nesse sentido. Será também exposto o funcionamento do \acrshort{tcg} para melhor entendimento de como podem ser implementadas as funcionalidades e a \acrshort{api} do lado da plataforma a desenvolver. A integração do \acrshort{tcg} neste projeto, foi uma escolha da empresa (i. e. cliente), e por isso não serão analisadas plataformas concorrentes do \acrshort{tcg} na medida em que esse trabalho é feito pela equipa do \acrshort{tcg} e apenas serão feitas as mudanças necessárias no mesmo para o desenvolvimento da \acrshort{api} de comunicação. Por fim serão explorados e analisados algoritmos de decisão para a criação de questionários, tal como questionários online já existentes, que implementam algoritmos de decisão.

Após a apresentação destas ferramentas será feita uma análise das vantagens e desvantagens de cada uma, assim como a comparação de funcionalidades.

REVER A INTRODUÇÃO TODA

\section{Formulários}
\label{formulários}

\subsection{SurveyMonkey}
\label{surveyMonkeyM}

O SurveyMonkey é uma plataforma \acrfull{saas} de criação de formulários online. Permite recolher informações do público alvo através de formulários e personalizar, segmentar e visualizar esta informação.



\subsection{Typeform}
\label{typeformM}

O Typeform é uma plataforma \acrshort{saas} de criação de formulários online. É uma empresa que afirma resolver o problema dos formulários e inquéritos aborrecidos e tem também como proposta de valor o facto de conseguir criar formulários e inquéritos sem ter que programar uma única linhad e código. Esta plataforma permite recolher informações do público alvo através de formulários e inquéritos personalizados e no final visualizar estes dados. 



\subsection{Google Form}
\label{googleformM}

O Google Form é uma aplicação de adminstração de inquéritos que está incluída no Google Drive office juntamente com o Google Docs\cite{gdocs}, Google Sheets e Google Slides\cite{gslides}. Esta ferramenta permite recolher informações do público alvo através de formulários e inquéritos personalizados e automaticamente exportar os dados para uma \textit{google sheet}.



\section{\acrfull{tcg}}
\label{sec:TCGM}

O \acrlong{tcg} é um produto actualmente no mercado, desenvolvido pela equipa da 10.digital, que tem como principal objectivo transformar PDFs numa aprendizagem baseada em tentativa erro.

 O \acrshort{tcg} nasceu de uma forte convicção de que perder apenas 2 minutos por dia numa formação tentativa erro é uma optima forma de aprender, poupando tempo e dinheiro às empresas. Inicialmente muito focado em formação interna, a equipa do \acrshort{tcg} foi-se apercebendo que existem muitos outros problemas (e. g. Consolidação de procedimentos, \textit{Onboarding} de novos colaboradores, Divulgação da cultura da empresa, Divulgação de informações técnicas a parceiros/clientes ...) para o qual a plataforma tem solução (e. g. Assimilação da cultura de empresa e do espírito das marcas, Simplificação do processo de acolhimento, Redução de custos em reuniões periódicas, Facilidade em divulgar aspectos técnicos, que de outra, forma demorariam mais tempo ...).\cite{tcginfo}



\section{Questionários}
\label{questionarios}

\subsection{involve.me}
\label{involvemeM}

%Não aceita multiplos resultados por resposta
%não cria perfis de utilizador

"\textit{involve.me is a next-generation user engagement \& customer experience platform with a focus on digital marketers \& e-commerce.}"\cite{involve}.
Esta plataforma foca-se também em recolher e analisar informações sobre os utilizadores finais.

\subsection{}
\label{}





Um dos objectivos do projecto é possibilitar aos utilizadores da plataforma conseguirem criar questionários que, baseado nas respostas do utilizador final, apresente um resultado no fim do mesmo, tal como referido no Capítulo \ref{subsec:objetivos}.

O Akinator\cite{akinator} e o high5test\cite{5} são dois bons exemplos de aplicações que através de um questionário e baseado nas respostas do utilizador final, apresenta um resultado. Estas duas aplicações são bastante poderosas e utilizam algoritmos de decisão que os permitem tirar este tipo de conclusões. Nesta medida alguns algoritmos como o CART\cite{cart}, ID3\cite{id3}\cite{id3_2}\cite{cart} e C4.5\cite{cart}\cite{c4.5} foram brevemente estudados com intuito de serem aplicados nesta funcionalidade do projecto. Este tipo de algoritmos, não só são complexos como são algoritmos que necessitam de dados de treino. Tal como referido no Capítulo \ref{subsec:objetivos}, um dos requisitos definidos pelo cliente é que a criação destes questionário seja intuitiva e exequível por qualquer pessoa mesmo que esta não tenha quaisquer conhecimento em programação. Digo isto e somando o facto de que, em qualquer momento da criação dos questionário, não haverá dados de treino, a aplicação deste tipo de algoritmos não será possível.

Dito isto foi pensado outras abordagens e determinou-se que um resultado semelhante (i. e. satisfazendo as necessidades do cliente) consegue ser alcançado com um sistema de pontuações. Nesta abordagem são atribuidos pesos a cada pergunta e consoante a resposta do utilizador o sistam vai pontuando os resultados possíveis para que no final possa ser apresentado o resultado com maior pontuação.

Uma aplicação deste típo de questionário será, por exemplo: "Calcule o lugar ideal de para viajar nas suas férias". Actualmente, num formato semelhante (i. e. em formato \textit{quiz}) já existem alguns \textit{Websites}(e. g. Chase for Adventure\cite{chaseforadventure}, travelpicker\cite{travelpicker}, Insight Vacations\cite{insightvacations} e Driftwood Journals\cite{driftwoodjournals}) que satisfazem, de forma simplista, esta necessidade em especifico. Apesar de simples são exemplos que se conseguiriam replicar com a plataforma a desenvolver e que satisfazem as necessidades do cliente. 

\section{Concursos}
https://www.shortstack.com/

https://www.wishpond.com/

https://www.tryinteract.com/

\section{Discussão de funcionalidades}
\label{comparacao}




%STATS
% Overall stats




















Nas Tabelas \ref{tab:comparacao1} e \ref{tab:comparacao} encontra-se a comparação entre as ferramentas analisadas nos secções \ref{surveyMonkey}, \ref{typeform} e \ref{googleform}, baseada numa lista de funcionalidades.

Como vimos anteriormente em todas as plataformas/ferramentas é necessário criar uma conta para aceder a todas as funcionalides e em todas as plataformas analisadas é possível criar conta e iniciar sessão através de sistemas externos (\acrshort{api}). O Projeto a desenvolver segue um modelo \gls{b2b} e por isso mesmo não é de grande importancia implementar esse tipo de funcionalidades pelo que, devido à sua relevância, não foi referido na Tabela \ref{tab:comparacao}.

O plataforma da Google fornece, no plano gratuito, todas as funcionalidades, ao contrário de todas as outras ferramentas analisadas anteriormente, que, tal como se passará com a plataforma a desenvolver, para se ter acesso a todas as funcionalidades, ou pacotes de funcionalidades, terá de ser paga uma subscrição. \textcolor{red}{DEMO? a definir com o cliente ou com quem trata dessas coisas}

Todas as platafomas permitem a criação de formulários do zero, e a plataforma da 10.digital não é excepção. Tal como foi definido na estratégia de negócio, os conteúdos que serão lançados nas formações, questionários e concursos não serão da autoridade da 10.digital, a menos que estejamos incluídos em algum projeto relacionado. Dito isto facilmente se decidiu que a plataforma a desenvolver terá, tal como todas as outras ferramentas analisadas, a funcionalidade de poder adicionar conteúdo previamente feito de forma rápido (i. e. adicionar um ficheiro com todo o conteúdo estruturado). 


Tal como foi referido no capítulo \ref{sec:introducao}, secção \ref{subsec:contexto}, o inbound é uma estratégia de marketing que se foca em criar razões para os publico alvo vir até nós através da criação de conteúdo interessante, útil, relevante etc... Para manter esta procura por parte dos clientes é necessário haver valor ao longo da jornada e ideialmente proporcionar uma boa experiência ao utilizador. Nesta medida a personalização das formações, questionários e concursos é muito importante tanto a nível de conteúdo como estético e funcional para que o utilizador se sinta valorizado. Para tornar isto possível, tal como o SurveyMonkey e o Typoform, a plataforma a desenvolver será a algumas funcionalidades que lhe permitirá criar vários tipos de pergunta e estéticamente melhorar a experiência do utilizador final.


%Antes de enviar/partilhar um formulário é sempre importante pré-visualizar e testar. Neste aspecto a platforma a desenvolver não é diferente e em relação ao envio de formulários, ao contrário de todas as outras ferramentas será possível definir uma rotina, podendo enviar formulários todos os dias, semanas ou meses, a uma hora a definir.

Por fim temos a analise e tratamento de dados que é um dos suportes daquilo que é o tripé do marketing digital. Como será de esperar a plataforma a desenvolver, tal como todas as restantes plataformas analisadas, não é um software dedicado à analise e tratamento de dados, na medida que terá limitações, contudo, ao contrário do Typeform e do Google Form, terá as funcionalidades necessárias para satisfazer as necessidades do utilizador. Analisando mais em detalhe a plataforma SurveyMonkey, que das ferramentas analisadadas, foi a única que apresentou capacidade de filtrar e segmentar os dados recolhidos, será necessário perceber que funcionalidades podemos melhorar e o que podemos acrescentar. Outro aspecto onde a plataforma da 10.digital se pode destacar será na possibilidade da criação de perfis de utilizador através da utilização de \textit{tags}. 
\textcolor{red}{ FALTA FALAR NA PARTILHA DE DADOS}

\textcolor{red}{
10.digital dá para dar feedback a cada pergunta.
falar que não analisei a concorrencia do tcg porque esse trabalho já foi feito pela equipa do TCG.
}




\textcolor{red}{TABELA POR ACTUALIZAR. ACTUALIZAR ASSIM QUE ACABAR A ANALISE DAS FERRAMENTAS DE CRIAÇÃO DE CONCURSOS ONLINE}
	\renewcommand{\arraystretch}{2.5}
\setlength\arrayrulewidth{1.5pt}
\begin{table}[!ht]  
	\begin{center}
	\begin{tabular}{|p{4cm}|p{1.5cm}|p{1.5cm}|p{1.5cm}|p{1.5cm}|}
		\cline{2-5}
		\multicolumn{1}{c|}{} & \hspace{0.6cm}\begin{sideways}SurveyMonkey.\end{sideways} & \hspace{0.6cm}\begin{sideways}Typeform\end{sideways} & \hspace{0.6cm}\begin{sideways}Google Form\end{sideways} &\hspace{0.6cm}\begin{sideways} 10.digital\end{sideways}\\ \hline
		
		
			Plano Gratuito & \cellcolor{yellow!80}   & \cellcolor{yellow!80}  & \cellcolor{green!80} & \cellcolor{yellow!80}  \\ \hline
		
		Criar Formulário do zero & \cellcolor{green!80}  & \cellcolor{green!80}  & \cellcolor{green!80} & \cellcolor{green!80} \\ \hline
		
		Templates de formulários disponíveis& \cellcolor{green!80}  & \cellcolor{green!80} & \cellcolor{green!80} & ???? \\ \hline
		
		Adicionar conteúdo previamente feitos & \cellcolor{green!80}   & \cellcolor{red!80}  & \cellcolor{green!80} & \cellcolor{green!80}  \\ \hline
		
		Tipos de perguntas & \cellcolor{green!80}  & \cellcolor{green!80}  & \cellcolor{yellow!80} & \cellcolor{green!80}  \\ \hline
		
					
	\end{tabular}
\end{center}
		\hspace{1.2cm}	\textcolor{red}{$\blacksquare$} Funcionalidade não implementada
		
	   \hspace{1.2cm}     \textcolor{yellow}{$\blacksquare$} Funcionalidade parcialmente implementada
	   
	    \hspace{1.2cm}     \textcolor{green}{$\blacksquare$} Funcionalidade totalmente implementada 
	   \begin{center}
\caption{Tabela de comparações de funcionalidades}
\label{tab:comparacao1}
\end{center}
\end{table}

\newpage
		

		\renewcommand{\arraystretch}{2.5}
		\setlength\arrayrulewidth{1.5pt}
	\begin{table}[!ht]  
		\begin{center}
		\begin{tabular}{|p{4cm}|p{1.5cm}|p{1.5cm}|p{1.5cm}|p{1.5cm}|}
			\cline{2-5}
			\multicolumn{1}{c|}{} & \hspace{0.6cm}\begin{sideways}SurveyMonkey.\end{sideways} & \hspace{0.6cm}\begin{sideways}Typeform\end{sideways} & \hspace{0.6cm}\begin{sideways}Google Form\end{sideways} &\hspace{0.6cm}\begin{sideways} 10.digital\end{sideways}\\ \hline
			
		
				\textit{Drag and Drop} & \cellcolor{green!80}   & \cellcolor{red!80}  & \cellcolor{red!80} & \cellcolor{red!80} \\ \hline
				
			% Recomendações de perguntas& \cellcolor{green!80}  & \cellcolor{green!80}  & \cellcolor{blue!25} & \cellcolor{blue!25}  \\ \hline
			Personalização do formulário& \cellcolor{green!80}    & \cellcolor{green!80}   & \cellcolor{yellow!80} & \cellcolor{green!80}   \\ \hline
			
			 Pré-visualização do formulário& \cellcolor{green!80}  & \cellcolor{green!80}  & \cellcolor{green!80} & \cellcolor{green!80} \\ \hline
			 
			 	Deixar \textit{feedback} sobre as perguntas do formulário& \cellcolor{red!80}    & \cellcolor{red!80}   & \cellcolor{red!80} & \cellcolor{green!80}   \\ \hline
			
			 Integração de sistemas externos& \cellcolor{red!80}   & \cellcolor{green!80} & \cellcolor{yellow!80}  & ????  \\ \hline
			
			Envio do formulário de forma periódica & \cellcolor{red!80}   & \cellcolor{red!80}  & \cellcolor{red!80} & \cellcolor{green!80} \\ \hline
			
			 Analise de resultados & \cellcolor{green!80}   & \cellcolor{yellow!80} & \cellcolor{yellow!80} & \cellcolor{green!80}  \\ \hline
			
			Partilha dos resultados & \cellcolor{green!80}   & \cellcolor{green!80}   & \cellcolor{green!80}  & ???? \\ \hline
			
			Exportar os resultados & \cellcolor{green!80}   & \cellcolor{green!80}   & \cellcolor{green!80}  & \cellcolor{green!80}  \\ \hline
			
			
		\end{tabular}
	\end{center}
\hspace{1.2cm}	\textcolor{red}{$\blacksquare$} Funcionalidade não implementada

\hspace{1.2cm}     \textcolor{yellow}{$\blacksquare$} Funcionalidade parcialmente implementada

\hspace{1.2cm}     \textcolor{green}{$\blacksquare$} Funcionalidade totalmente implementada 
\begin{center}
	\caption{Tabela de comparações de funcionalidades}
	\label{tab:comparacao}
\end{center}
	\end{table}


%-------------------------------------------------------------------------------------------------
\blankpage
%-------------------------------------------------------------------------------------------------

\glsresetall



