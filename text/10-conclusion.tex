\chapter{Conclusion}
\label{sec:conclusion}


Este documento reflete o trabalho realizado durante o primeiro semestre. Esta foi uma fase de muita pesquisa e documentação, para facilitar as fases posteriores do projeto. 

Durante estes ultimos quatro meses o trabalho incidiu principalmente no estudo de mercado, planeamento, escolha da metodologia de trabalho, especificação de requisitos e especificação da arquitectura.

O estudo de mercado, ou por outras palavras, o estado de arte, permitiu avaliar produtos concorrentes e produtos que apesar de terem propósitos distintos, partilham as mesmas funcionalidades. Este estudo permitiu avaliar os pontos fortes e fracos das plataformas que podem ser concorrentes do 10.quest. Desta forma, este estudo não só ajudou a entender a posição do produto em relação ao mercado mas também ajudou a perceber como melhorar a mesma.

A metodologia de trabalho adoptada foi a metodologia utilizada pela empresa. Esta metodologia ajudou na fase da especificação de requisitos visto que, a meio do semestre surgiram novas funcionalidades para o projeto e rápidamente houve uma adaptação.

A Arquitectura ajudou a realizar algumas decisões, apesar de grande parte das decisões ainda estarem por tomar devido a migração de técnologias na empresa, e ajudou também encontrar alguns riscos que podem vir a surgir durante o desenvolvimento. A antecipação destes riscos ajudou na criação de planos de mitigação e assim evitar algumas barreiras no futuro. 

Durante este primeiro projeto surgiraram algumas dificuldades e foram cometidos alguns erros que provocaram alguns atrasos, contudo estas barreiras foram superadas e resolvidas da melhor forma possível. 




%-------------------------------------------------------------------------------------------------
\blankpage
%-------------------------------------------------------------------------------------------------

\glsresetall
