\chapter{Introdução}
\label{sec:introducao}

O presente documento expõe todo o trabalho desenvolvido ao longo do ano, no âmbito da unidade curricular, Estágio/Dissertação do Mestrado de Engenharia Informática. O estágio insere-se no ramo de Engenharia de Software e foi desenvolvido nas intalações da Dascat Software (10.Digital)\cite{10}, em empresa que foca as suas competências chaves em marketing digital. 

O estágio está a ser orientado pelo Professor Doutor Pedro Furtado, professor no Departamento de Engenharia Informática e pelo Engenheiro Pedro Beck, \acrshort{cto} na 10.digital.

Este capítulo está organizado em 4 secções. Em primeiro lugar é descrito o contexto em que o projecto se enquadra. A segunda e terceira secção explica o motivo para a realização deste projecto e quais os seus objectivos, respetivamente. Por fim é feita uma descrição da estrutura escolhida para este documento.

\section{Contexto}
\label{subsec:contexto}

Desde os primeiros dias da internet que várias técnicas de marketing têm sido praticadas, e ao longo dos anos estas técnicas têm evoluido. "For the first time the term “Inbound marketing” was used by Brian Halligan in 2005 (Halligan and Shah, 2009; Pollit, 2011)"\cite{bookinbound}\cite{inbound_paper}. Segundo uma publicação de Nextiny Marketing\cite{postNextiny}, as técnicas de inbound marketing começaram a aparecer de forma tímida na internet por volta de 2007 e foi em 2012 que começaram com um crescimento significativo.


\section{Motivação}
\label{subsec:motivacao}


\section{Objetivos}
\label{subsec:objetivos}



\section{Estrutura do documento}
\label{subsec:estrutura}

Este relatório está dividido em 9 capítulos, organizados da seguinte forma:
\begin{itemize}
	 \item \textbf{Capítulo \ref{sec:estado-arte} - Estado de Arte: }Análise comparativa das soluções já existentes para recolha de dados em estratégias de inbound marketing.
	
	\item \textbf{Capítulo \ref{sec:abordagem} - Abordagem: }Abordagens e decisões tomadas antes de definir o conceito do produto, metologias utilizadas e, de forma superficial, o planeamento das fases do desenvolvimento do projecto.
	
	\item \textbf{Capítulo \ref{sec:requisitos} - Especificação de Requisitos: }Levantamento, análise e documentação dos requisitos funcionais e não funcionais.
	
	\item \textbf{Capítulo \ref{sec:arquitetura} - Arquitetura: }Especificações da arquitetura, tecnologias a utilizar, desafios e análise de riscos.
	
	\item \textbf{Capítulo \ref{sec:implementacao} - Implementação: }Organização das funcionalidades implementadas por Sprints e descrição do processo de desenvolvimento.
	
	\item \textbf{Capítulo \ref{sec:validacao} - Validação: }Testes detalhados realizados para garantir a validação do sistema.
	
	\item \textbf{Capítulo \ref{sec:produto-final} - Produto Final: }Resultado final da plataforma.

	\item \textbf{Capítulo \ref{sec:conclusion} - Conclusão: }Conclusões referentes ao trabalho realizado.
	
\end{itemize}



%-------------------------------------------------------------------------------------------------
\blankpage
%-------------------------------------------------------------------------------------------------

\glsresetall



