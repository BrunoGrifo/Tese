\chapter{Introdução}
\label{sec:introducao}

O presente documento expõe todo o trabalho desenvolvido ao longo do ano, no âmbito da unidade curricular Estágio/Dissertação do Mestrado de Engenharia Informática. O estágio insere-se no ramo de Engenharia de Software e foi desenvolvido nas instalações da Dascat Software (10.digital)\cite{10}, empresa que foca as suas competências chaves em marketing e transformação digital. 

O estágio está a ser orientado pelo Professor Doutor Pedro Furtado, professor no Departamento de Engenharia Informática e pelo Engenheiro Pedro Beck, \acrshort{cto} na 10.digital.

Este capítulo está organizado em 4 secções. Em primeiro lugar é descrito o contexto em que o projeto se enquadra. A segunda e terceira secções explicam o motivo para a realização deste projeto e quais os seus objetivos, respetivamente. Por fim é feita uma descrição da estrutura escolhida para este documento.

\section{Contexto}
\label{subsec:contexto}

Desde os primeiros dias da internet que várias técnicas de marketing têm sido praticadas, tendo-se verificado uma evolução ao longo dos anos. "For the first time the term <<Inbound marketing>> was used by Brian Halligan in 2005 (Halligan and Shah, 2009; Pollit, 2011)"\cite{bookinbound}\cite{inbound_paper}. Segundo uma publicação de Nextiny Marketing\cite{postNextiny}, as técnicas de inbound marketing começaram a aparecer de forma tímida na internet por volta de 2007 e foi em 2012 que começaram com um crescimento significativo. A Internet foi uma das portas para a evolução das técnicas de marketing que conhecemos nos dias de hoje. As empresas passaram assim a conseguir apresentar-se e apresentar os seus produtos, digitais e não digitais, através de plataformas cada vez mais inovadoras e com novas funcionalidades como \textit{updates} rápidos, pagamentos \textit{online}, entrega imediata (i. e. produtos digitais), comunicação com os visitantes, formulários, questionários, jogos, entre outros.  

\textit{Inbound Marketing} é uma técnica de marketing que se foca em criar razões para o público alvo chegar até às empresas, através da criação de conteúdo interessante, relevante, e que, acima de tudo, dê valor ao visitante. Para continuar com esta procura é necessário manter o valor ao longo da jornada entre a empresa e a \textit{\gls{lead}}, que idealmente se tornará consumidor/cliente. "... Shah’s [Dharmesh Shah um dos pioneiros do Inbound Marketing] blog for startups, while he was still a graduate student, managed to attract more visitors than websites owned by companies with professional marketing teams and large budgets. Their conclusion was that <<People did not want to be interrupted by marketers or harassed by salespeople. They wanted to be helped>>."\cite{inbound_paper} . A ideia por detrás do inbound marketing é utilizar estratégias de marketing que tentam conquistar o interesse das \textit{\gls{lead}}s atraindo-as para as empresas (e. g. website, blog ...) em vez do método tradicional de enviar publicidade para os mesmos e esperar pela sua ação. Esta procura é captada através de conteúdos (e. g. texto, imagens, vídeos, livros digitais, \textit{how-to guides}) que acima de tudo adicionam valor às \textit{\gls{lead}}s.

De acordo com a plataforma \textit{HubSpot}\cite{HubSpot}, a metodologia de inbound marketing consiste em 3 fases: \textit{attack}, \textit{engage} e \textit{delight}. Na primeira fase, os possíveis clientes pesquisam informações \textit{online}, têm perguntas e problemas que querem resolver, sendo que, tipicamente vão a um motor de pesquisa, procurar soluções para o seu problema. Se, por exemplo, o blog da empresa der respostas às perguntas eles vão encontrar o post e tornar-se um visitante no nosso website.
 "When someone picks up your marketing materials you have 30 seconds or less to convince him or her that you can help. The Buyer is saying, "You better tell me something from the beginning that blows my mind, changes my world, or makes me say, 'Where have you been all my life?'"\cite{pbr}. 
A segunda fase da metodologia começa assim que o visitante ou \textit{lead} não qualificada, tem contacto com o website, blog, etc...O principal objectivo é converter as \textit{leads} não qualificadas em \textit{leads} qualificadas, recolhendo a sua informação, tipicamente, através de formulários. Visto que os visitantes ou \textit{leads} não fornecem as suas informações pessoais facilmente, eles têm de receber algo com valor em troca (e. g. livros digitais, software gratuito, tutoriais). 
Na terceira fase temos o \textit{delight}, que de acordo com a metodologia de inbound marketing, todas as empresas devem continuar a proporcionar uma boa experiência mesmo depois das \textit{leads} quialificadas se tornarem clientes. O esforço não pára assim que uma \textit{lead} se torna num cliente por isso, a empresa deve continuar a fornecer ao cliente conteúdos atraentes, qualidade de serviço superior e ao mesmo tempo ter em conta o seu \textit{feedback}. 

Se olharmos para o mercado como uma história, o público alvo está numa jornada que o vai levar a estados de consciêncialização, consideração e decisão. A empresa vai atuar como ponto de ligação entre eles e a sua marca.
É importante a empresa pôr-se no lugar do consumidor e tentar entender como é que a jornada dele se parece, desde a consciência do problema, consideração de possíveis soluções e decisão. A partir daí podemos criar conteúdos que consiga satisfazer as necessidades de todos os visitantes.
Antes de começar esta conexão, é necessário entender com quem têm de criar relações/ligações. O conteúdo que vai ser criado tem de ser relevante para o público e tem de começar a ser introduzido em sítios que o público alvo frequente para entrarmos no seu radar. 
Um dos elementos principais do inbound marketing é o website da empresa onde as \textit{leads} não qualificadas são convertidas em \textit{leads} qualificadas através da optimização de formulários e questionários que tipicamente oferecem conteúdo em troca de informações sobre a \textit{lead}. 
Nesta fase não basta a empresa focar-se nas informações que já têm sobre cada \textit{lead} ou cliente. É importante tentar conhecer mais sobre eles através de conteúdos que vão sendo fornecidos. Assim pode-se começar a contextualizar, segmentar e personalizar os conteúdos (e. g. emails). Desde logo cada email, mensagem, etc ... torna-se mais pessoal e melhora a experiência com o consumidor e não há nada melhor para que uma \textit{lead} ou cliente se sinta valorizado.
Idealmente as \textit{leads} serão convertidos em clientes e será criada uma relação, que com o tempo será longa e de confiança.



\section{Motivação}
\label{subsec:motivacao}

Num mundo em que o marketing decorre cada vez mais em cada ação que se realiza \textit{online}, conceitos como inbound marketing e \textit{\gls{social selling}} emergem, mais do que como tendências, como certezas e bases para um bom processo de marketing e vendas.

Hoje, várias empresas criam estratégias que permitem identificar e conhecer cada vez melhor os seus potenciais clientes. É comum observarem-se ações que permitem às pessoas verem vídeos, fazerem download de PDFs ou até terem acesso a \textit{\gls{webinars}} ou a \textit{streaming} de eventos em troca de um simples email. Também é cada vez mais comum vermos profissionais independentes a criarem cursos e enriquecerem através da venda dos mesmos.
Mas estas ações esbarram cada vez mais em dois problemas principais: já quase todas as empresas fazem o mesmo. E fruto disso, o tempo para consumir qualquer conteúdo é cada vez menor.

\begin{displayquote}
"Ela é mais um sistema de inbound marketing que pode ser integrado numa estratégia de \textit{\gls{social selling}}. Eu, profissional independente, faço um curso sobre LinkedIn\cite{linkedin} em muito menos tempo do que faria através de um vídeo e ensino as pessoas fazendo-as investir apenas 2 minutos por dia a consumir o meu conteúdo. Com isso, fico a saber quem são as pessoas interessadas em saber mais sobre LinkedIn e que poderão estar interessadas na versão \textit{premium} do meu curso, que tem um custo bem mais elevado.

Isto acontece com profissionais independentes, mas também com empresas. Eu, CEO da 10.digital, tenho cursos sobre estratégia digital. As pessoas que os subscrevem são potenciais interessados na minha empresa que vende estratégia digital: seja por serem potenciais clientes, seja por serem concorrentes, seja por serem potenciais colaboradores.

Desta forma, investindo pouco tempo a criar conteúdos, consigo manter uma forma de contactar com as pessoas diariamente durante semanas, expondo-lhes a minha marca e analisando o que sabem ou não sobre o assunto.
Posso propor-lhes outros conteúdos ou vender-lhes os meus serviços. A plataforma que estamos a desenvolver junta, assim, vários conceitos diferentes que podem ajudar a mudar o conhecimento que as empresas têm do mercado e a nutrir as leads, aumentando a sua capacidade de angariação de clientes com custos mais baixos e um método de ensino inovador para os seus clientes." 

- Pedro Girão, \acrfull{ceo} na 10.digital.
\end{displayquote}

\section{Objetivos}
\label{subsec:objetivos}

Os objetivos do projeto de estágio foram definidos pelo cliente, 10.digital, sendo que objetivo incide no desenvolvimento de uma plataforma de inbound marketing, que segue um modelo \textit{\gls{b2b}}. Esta plataforma tem como foco principal permitir ao \gls{user} criar e partilhar três tipos de campanhas: questionários de personalidade, concursos e formações, seguindo estratégias de inbound marketing. Os dados recolhidos através das campanhas são analisados e apresentados utilizando relatórios de dados que permitem ao utilizador da plataforma criar perfis de personalidade, perceber tendências de mercado e extrair informação essencial para a criação e otimização de novas outras campanhas para maximizar a geração \textit{leads}. O modelo de negócio da plataforma pode ser observado no Anexo \ref{a:bvm}. Nesta plataforma terá de ser desenvolvido um \gls{backoffice} que, de forma intuitiva:
\begin{itemize}
	\item[--] Permita criar questionários de personalidade que, baseado nas respostas do \gls{finaluser} (i. e. participante na campanha), recomende um resultado no fim do questionário de personalidade. Neste sentido não só terá de ser aplicado ou desenvolvido um algoritmo para cálculo do resultado final, descrito no Capítulo \ref{sec:requisitos}, como também terá de ser pensada uma forma intuitiva de o \gls{user} do \gls{backoffice} conseguir construir este mesmo questionário de personalidade, sem que seja necessário quaisquer conhecimentos de programação por parte do mesmo. Por outras palavras é uma forma intuitiva das empresas poderem promover os seus produtos, baseado nas preferências dos possíveis consumidores.
	\item[--] Permita criar formações baseadas numa aprendizagem por tentativa erro. \acrfull{tcg}\cite{tcg}, atualmente no mercado, é um produto desenvolvido pela 10.digital que tem este propósito (i. e. criação de formações baseadas numa aprendizagem por tentativa erro), tal como será descrito no Capítulo \ref{sec:estado-arte}. Neste aspeto será desenvolvida uma \acrshort{api} do lado do \acrshort{tcg} que irá disponibilizar as suas funcionalidades através de pedidos \acrshort{https}/\acrshort{rest} a outras aplicações. Desta forma a plataforma a desenvolver terá acesso às funcionalidades necessárias para criar formações, e aceder e tratar os dados recolhidos pelas mesmas.	
	\item[--] Permita criar concursos em formato de \textit{quiz}. As pessoas interessam-se, todos os dias, por milhares e milhares de eventos que decorrem nas mais diversas áreas (e. g. cinema, futebol, ténis, teatro, festivais, concertos, corridas de motos e automóveis, exposições, \textit{Web Summits}\cite{websummit}, visitas a museus, viagens...) e para além disso
	vemos milhões de pessoas com despesas mensais fixas relacionadas, ou não, com hobbies (i. e. a subscrição mensal para jogar padel, para ir ao ginásio, para ir à piscina, à ioga e inumeras outras coisas como meter gasolina etc...). Neste sentido terá de ser possível criar concursos em forma de \textit{quiz}, em que as pessoas pagam uma pequena quantia, relativamente ao prémio que podem ganhar (e. g. "Mostra que sabes tudo sobre tecnologia e ganha 1 bilhete para o Web Summit"). No final de cada concurso, os participantes terão de receber a sua classificação, tal como os resultados das suas respostas.
	\item[--] Permita ao \gls{finaluser} participar nas campanhas referidas nos pontos anteriores. O \gls{finaluser} deve conseguir inscrever-se numa campanha através de uma \textit{landing page}, criada pela autor da campanha (i. e. \gls{user} da plataforma/\gls{backoffice}), e de seguida receber por email o \textit{link} de acesso à campanha. Caso o \gls{finaluser} se inscreva numa campanha do tipo questionário de personalidade, o \gls{finaluser}  deve poder participar respondendo às perguntas, e no final, baseando-se nas respostas do \gls{finaluser}, a plataforma 10.quest deve recomendar um resultado. Caso o \gls{finaluser} se inscreva numa campanha do tipo concurso, o mesmo deve poder efectuar um pagamento para poder participar, e no final da campanha, deve receber um mail com a sua classificação e as informações do prémio se for o caso. Por fim se o \gls{finaluser} se inscreva numa campanha do tipo formação, deve receber periodicamente no email, um link para a formação. Durante a formação, o \gls{finaluser}, no fim de responder a todas as perguntas deve conseguir ver o seu desempenho (i. e. número de respostas certas e erradas, participação etc..) desde o inicio da formação.
	\item[--] Permita exibir os dados recolhidos, através de um relatório de dados, permitindo ao \gls{user} do \gls{backoffice} interpretar de forma fácil os resultados. Cada tipo de campanha irá recolher direfentes tipos de dados adicionais e neste sentido os relatórios terão de se adaptar a cada tipo de campanha. Relativamente às formações, os dados serão recolhidos pelo TCG e por consequente, todos os dados relativamente a participação, percentagem de sucesso, perguntas mais dificeis, entre outros, que estão detalhados na secção \ref{sec:TCG}, terão de ser acedidos através da \acrshort{api} a desenvolver. Relativamente aos questionários de personalidade, os dados recolhidos irão permitir mostrar tendências de mercado, níveis de participação, tabela de leads onde se pode ver os perfis de personalidade entre outros, que estão mais detalhados na secção \ref{d:relatorios}. Relativamente aos concursos terá de ser representada a tabela de classificações. Vários dados recolhidos e processados são comuns aos vários tipos de campanhas e por isso mesmo serão representados no relatório geral. Neste relatório terá de ser representado o gráfico que mostra o tráfego em todas as campanhas e uma tabela de todas as \textit{leads} geradas pelas campanhas onde será possivel segmentar e criar perfil de personalidade através de \gls{tags}.
\end{itemize}



\section{Estrutura do documento}
\label{subsec:estrutura}

Este relatório está dividido em 9 capítulos, organizados da seguinte forma:
\begin{itemize}
	 \item \textbf{Capítulo \ref{sec:estado-arte} - Estado de Arte: }Análise comparativa das soluções já existentes para recolha de dados em estratégias de inbound marketing.
	
	\item \textbf{Capítulo \ref{sec:abordagem} - Abordagem: }Metodologia, análise de riscos e planeamento das fases do desenvolvimento do projeto.
	
	\item \textbf{Capítulo \ref{sec:requisitos} - Especificação de Requisitos: }Levantamento, análise e documentação dos requisitos funcionais e não funcionais.
	
	\item \textbf{Capítulo \ref{sec:gdpr} - Política de Privacidade: }Termos de serviço e política de privacidade do produto.
	
	\item \textbf{Capítulo \ref{sec:arquitetura} - Arquitetura: }Especificações da arquitetura, tecnologias a utilizar, desafios e análise de riscos.
	
	
	\item \textbf{Capítulo \ref{sec:implementacao} - Implementação: }Organização das funcionalidades implementadas por Sprints e descrição do processo de desenvolvimento.
	
	\item \textbf{Capítulo \ref{sec:validacao} - Validação: }Práticas e testes detalhados, realizados, para garantir a validação do sistema.
	
	\item \textbf{Capítulo \ref{sec:produto-final} - Produto Final: }Resultado final da plataforma.

	\item \textbf{Capítulo \ref{sec:conclusion} - Conclusão: }Conclusões referentes ao trabalho realizado.
	
\end{itemize}



%-------------------------------------------------------------------------------------------------
\blankpage
%-------------------------------------------------------------------------------------------------

\glsresetall



